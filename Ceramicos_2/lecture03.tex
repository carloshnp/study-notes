\lecture{0.3}{}{Moagem e Conformação}

\section*{Prensagem}

A prensagem é realizada colocando o pó, pré-misturado com ligantes e lubrificantes adequados, e pré-consolidado de modo que seja de fácil fluxo, em uma matriz e aplicando pressão para obter compactação. Duas categorias de prensagem são comumente utilizadas: (1) uniaxial e (2) isostática. Ambas usam pó preparado pelos mesmos procedimentos.

\subsection*{Passos na prensagem}

Dividiremos em dois procedimentos, A e B.

\paragraph*{Procedimento A}

O Procedimento A é baseado em granulação para alcançar um pó de fluxo livre. As matérias-primas são selecionadas e pesadas de acordo com o cálculo adequado de batelada. Os pós são dimensionados por moagem a seco, sendo colocados em um misturador de muller com adições do ligante + cerca de 15\% em peso de água, e é misturado até ficar homogêneo.

A mistura é transformada em grânulos por peneiramento, passagem por um granulador, ou pré-prensagem + granulação. Os grânulos estão macios e úmidos nesta fase, se tornando moderadamente duros após a secagem. Os grânulos secos são classificados por peneiramento para alcançar a distribuição de tamanho desejada. Isso geralmente envolve a remoção de grãos finos que não estão fluindo livremente.

Neste ponto, uma verificação de qualidade em processo é realizada. Isso geralmente envolve a prensagem de uma amostra de teste para determinar as características de compactação (por exemplo, taxa de compactação de densidade verde, facilidade de liberação da matriz) e características de densificação (por exemplo, contração, densidade queimada e propriedades-chave). O pó aceitável está então pronto para ser preparado para a prensagem de produção. Isso pode envolver a adição de um lubrificante e um pouco de umidade e remoção por separação magnética de partículas metálicas recolhidas pelo desgaste do equipamento de processamento.

A taxa de compactação é a razão entre a espessura do pó na matriz e a espessura após a prensagem. Densidade verde é a densidade a granel do compacto. O termo "verde" é comumente usado para descrever o compacto cerâmico poroso antes da densificação.

\paragraph*{Procedimento B}

O Procedimento B é baseado em secagem por pulverização para alcançar um pó de fluxo livre. A batelada pesada de pó + aditivos é misturada com água suficiente para formar uma suspensão fluida (lodo) e moído úmido para obter uma mistura homogênea e dimensionamento de partículas.

O lodo é passado por uma peneira e/ou um separador magnético para remover partículas grandes e contaminação metálica. O lodo é então submetido a secagem por pulverização. Dependendo do lodo e dos parâmetros de secagem por pulverização, o pó resultante pode consistir em esferas sólidas, esferas ocas ou plaquetas em forma de rosquinha. Após a secagem por pulverização, o pó passa pelos mesmos procedimentos de controle de qualidade e prensagem conforme descrito para o Procedimento A.

\subsection*{Seleção de aditivos}

Os aditivos comumente necessários para a prensagem são um ligante, um plastificante, um lubrificante e/ou um auxiliar de compactação. O ligante fornece alguma lubrificação durante a prensagem, e confere à peça prensada resistência adequada para manuseio, inspeção e usinagem verde. O plastificante modifica o ligante para torná-lo mais maleável. O lubrificante reduz o atrito entre as partículas e o atrito nas paredes da matriz. O auxiliar de compactação (que é essencialmente um lubrificante) reduz o atrito entre as partículas.

Os efeitos combinados dos aditivos são: (1) permitir que as partículas de pó deslizem umas sobre as outras para se reorganizarem no empacotamento mais próximo possível, e (2) minimizar o atrito e permitir que todas as regiões do compacto recebam pressão equivalente.

\paragraph*{Ligantes e plastificantes}

A maioria dos ligantes e plastificantes são orgânicos. Eles revestem as partículas cerâmicas e fornecem lubrificação durante a prensagem e uma ligação temporária após a prensagem. A quantidade de ligante orgânico necessária para a prensagem é bastante baixa, geralmente variando de 0,5 a 5\% em peso. Os ligantes orgânicos normalmente são decompostos durante a etapa de densificação em alta temperatura e evoluem como gases. Alguns ligantes deixam um resíduo de carbono, especialmente se queimados em condições redutoras.

Existem também ligantes inorgânicos. Minerais argilosos, como a caulinita, são um bom exemplo. A caulinita possui uma estrutura em camadas e interage com a água para produzir uma mistura flexível e plástica. Os minerais argilosos não se queimam durante a densificação, mas, em vez disso, tornam-se parte do cerâmico.

A seleção do ligante depende do tipo de prensagem que será realizada. Alguns ligantes, como ceras e gomas, são muito macios e bastante sensíveis a variações de temperatura. Geralmente, esses não exigem adições de umidade ou lubrificante antes da prensagem, mas devem ser manuseados com mais cuidado para evitar alterações no tamanho dos grânulos que possam alterar as características de fluxo na matriz de prensagem ou resultar em distribuição de densidade inomogênea. Ligantes macios também têm tendência a extrudar entre os componentes da matriz, o que pode causar aderência ou reduzir a taxa de produção.

Outros ligantes podem ser classificados como duros; ou seja, produzem grânulos que são duros ou resistentes. Esses grânulos têm a vantagem de serem dimensionalmente estáveis e de fluxo livre, sendo, portanto, excelentes para produção em grande volume com prensas automatizadas. No entanto, geralmente não são auto lubrificantes e, portanto, exigem pequenas adições de lubrificante e umidade antes da prensagem. Eles também requerem pressão mais alta para garantir compactos uniformes. Se os aglomerados de pó inicial não formarem completamente um compacto contínuo durante a prensagem, artefatos com tamanho aproximado dos aglomerados persistirão nas etapas restantes do processo e podem atuar como falhas grandes, o que limitará a resistência.

Dextrina, amidos, ligninas e acrilatos produzem grânulos relativamente duros. Álcool polivinílico (PVA) e metilcelulose resultam em grânulos ligeiramente mais macios. Ceras, emulsões de cera e algumas gomas produzem grânulos macios.

A dureza e as características de deformação dos ligantes orgânicos variam com a temperatura, umidade e outros fatores. Muitos desses materiais passam por uma transição dúctil-frágil e se comportam de maneira frágil abaixo da transição e de maneira dúctil acima dela. A temperatura na qual essa transição dúctil-frágil ocorre é referida como temperatura de transição vítrea ($T_g$).

Vários aspectos das curvas de $T_g$ são importantes durante a prensagem: (1) a deformação total; (2) a quantidade de recuo ou recuperação após a remoção da carga; (3) a carga necessária para iniciar a deformação; e (4) a deformação líquida (deformação permanente).

Abaixo de Tg, a deformação é principalmente elástica e o comportamento é categorizado como \textit{vítreo}. A deformação total é baixa e é completamente recuperada após a remoção da carga. Esse comportamento oferece pouca ou nenhuma capacidade de ligante ou lubrificante. O material tem uma tendência mais forte a fraturar do que a se deformar. Por outro lado, acima de Tg, a deformação é grande e principalmente plástica. Muito pouco recuo ocorre quando a carga é removida. Esse comportamento oferece excelente capacidade de ligante e lubrificante. O comportamento intermediário é definido como \textit{viscoelástico}, e outros comportamentos são encontrados, como o comportamento \textit{borrachoso}, caracterizado pela alta deformação elástica e um grande recuo após a remoção da carga. Esse comportamento não é favorável para a prensagem. A completa faixa de características de deformação pode ocorrer para um único material de ligante orgânico ao longo de uma faixa de temperatura.


O comportamento de deformação pode ser alterado pela adição de plastificantes, que diminuem a $T_g$ e aumentam a densidade verde obtida durante a prensagem.

\paragraph*{Lubrificantes e auxiliares de compactação}

Lubrificantes e auxiliares de compactação são essencialmente iguais. Eles reduzem o atrito entre partículas, entre grânulos e entre o compacto de pó e a parede da matriz de prensagem. Isso resulta em uma maior uniformidade da peça prensada, melhor densidade verde, vida útil estendida da ferramenta, redução de aderência (o que reduz o tempo necessário para a limpeza da ferramenta) e diminuição da pressão necessária para ejetar a peça da matriz. Materiais com baixa resistência ao cisalhament são, geralmente, bons lubrificantes. Outros materiais de baixa resistência ao cisalhamento, além do estearato de zinco, têm sido bem-sucedidos como lubrificantes.

\paragraph*{Remoção de aditivos orgânicos}

A seleção do ligante e de outros aditivos deve ser compatível com a química do cerâmico e com os requisitos de pureza da aplicação. O ligante deve ser removido antes da densificação do cerâmico. Ligantes orgânicos podem ser removidos por decomposição térmica. Se ocorrer reação entre o ligante e o cerâmico abaixo da temperatura de decomposição do ligante ou se o cerâmico se densificar abaixo desta temperatura, a peça final será contaminada ou até mesmo poderá rachar ou inflar. Se a temperatura for elevada muito rapidamente ou se a atmosfera no forno for redutora, o ligante pode carbonizar em vez de se decompor, deixando resíduos de carbono.

\subsection*{Prensagem Uniaxial}

A prensagem uniaxial envolve a compactação de pó em uma matriz rígida aplicando pressão ao longo de uma única direção axial por meio de um êmbolo, pistão ou socador rígido. A maioria das prensas uniaxiais é mecânica ou hidráulica. As prensas mecânicas geralmente possuem uma taxa de produção mais alta e são fáceis de automatizar.

Os punções são posicionados antecipadamente no corpo da matriz para formar uma cavidade predeterminada (com base na taxa de compactação do pó) para conter o volume correto e alcançar as dimensões verdes necessárias após a compactação. A sapata de alimentação então se move para a posição e preenche a cavidade com pó de fluxo livre contendo ligantes adequados, umidade e lubrificante. A sapata de alimentação se retrai, alisando a superfície do pó à medida que passa, e os punções superiores descem para pré-comprimir o pó. Os punções superiores e inferiores comprimem simultaneamente o pó à medida que se movem independentemente para posições pré-determinadas. Os punções superiores se retraem, e os punções inferiores ejetam o compacto do corpo da matriz. A sapata de alimentação então se move para a posição e empurra o compacto para longe dos punções enquanto estes se ajustam para aceitar o preenchimento correto de pó. Este ciclo se repete geralmente de 6 a 100 vezes por minuto, dependendo da prensa e da forma que está sendo fabricada.

Outro tipo de prensa mecânica é a prensa rotativa. Inúmeras matrizes são colocadas em uma mesa giratória. Os punções da matriz passam por cames à medida que a mesa gira, resultando em um ciclo de preenchimento, compressão e ejeção semelhante ao descrito para uma prensa de curso único. Taxas de produção na faixa de 2000 peças por minuto podem ser alcançadas com uma prensa rotativa.

Outro tipo de prensa mecânica é a prensa de alavanca. É comumente usada para pressionar tijolos refratários e é capaz de exercer pressão de até cerca de 727.000 kg (800 t). A prensa de alavanca fecha em um volume definido para que a densidade final seja controlada em grande parte pelas características da alimentação.

Prensas hidráulicas transmitem pressão por meio de um fluido contra um pistão. Geralmente são operados com uma pressão definida, de modo que o tamanho e as características do componente prensado são determinados pela natureza da alimentação, a quantidade de preenchimento da matriz e a pressão aplicada. Prensas hidráulicas podem ser muito grandes, mas têm uma taxa de ciclo muito mais baixa do que as prensas mecânicas.

O tipo de prensa e ferramenta selecionados são baseados principalmente no tamanho e na forma da peça a ser prensada. As peças podem ser divididas em classes, como é feito na metalurgia do pó. As classes são definidas na Tabela 13.4. Peças com espessura constante e seção transversal fina podem ser pressionadas com sucesso com uma ação única, ou seja, com a matriz e o punção inferior estacionários e apenas o punção superior se movendo. Peças mais espessas não alcançam uma compactação uniforme se pressionadas apenas de uma extremidade. Essas exigem ferramentas em que os punções superior e inferior se movem, ou seja, ferramentas de dupla ação. Peças com variações na espessura da seção transversal requerem um punção independente para cada nível de espessura. Isso é necessário novamente para alcançar uma compactação uniforme em toda a peça. O punção só precisa percorrer uma distância A para alcançar a compactação da seção fina, mas deve percorrer uma distância A + B para compactar a seção espessa. Ambos não podem ser alcançados com um único punção rígido.

\includegraphics*[width=\linewidth]{./images/tabela_prensa_uniaxial.png}

\subsection*{Prensagem à seco}

A maioria das prensagens automatizadas é realizada com pó granulado ou secado por pulverização contendo de 0 a 4\% de umidade. Isso é referido como prensagem a seco, semiúmida ou a pó. A compactação ocorre por esmagamento dos grânulos e redistribuição mecânica das partículas em uma disposição de empacotamento próximo. O lubrificante e o ligante geralmente auxiliam nessa redistribuição, e o ligante fornece coesão. Pressões elevadas são normalmente usadas para a prensagem a seco para garantir a quebra dos grânulos e a compactação uniforme.

\subsection*{Prensagem a úmido}

A prensagem úmida envolve um pó de alimentação contendo 10 a 15\% de umidade e é frequentemente utilizado com composições contendo argila. Este pó de alimentação se deforma plasticamente durante a prensagem e se conforma ao contorno da cavidade da matriz. A forma prensada geralmente contém rebarbas (camadas finas de material nas bordas onde o material extrudado entre as partes da matriz) e pode se deformar após a prensagem se não for manuseada com cuidado. Por esses motivos, a prensagem úmida não é adequada para automação. Além disso, tolerâncias dimensionais geralmente são mantidas apenas em ±2%.

Os seguintes são alguns dos problemas que podem ser encontrados com a prensagem uniaxial:

\begin{itemize}
    \item Densidade ou tamanho inadequados
    \item Desgaste da matriz
    \item Rachaduras
    \item Variação de densidade
\end{itemize}

Os dois primeiros são fáceis de detectar por meio de medições simples no compacto verde imediatamente após a prensagem. Densidade ou tamanho inadequados estão frequentemente associados a lotes de pó fora das especificações e, portanto, são relativamente fáceis de resolver. O desgaste da matriz se manifesta como uma mudança progressiva nas dimensões. Também deve ser tratado rotineiramente pela especificação do processo e pelo controle de qualidade.

A origem das rachaduras pode ser mais difícil de localizar. Pode ser devido a um projeto incorreto da matriz, aprisionamento de ar, recuo durante a ejeção da matriz, atrito na parede da matriz, desgaste da matriz ou outras causas. Muitas vezes, uma rachadura se inicia na borda superior da peça durante a liberação da pressão ou ejeção da peça.

O primeiro, mostrado na Figura 13.13a, ocorre quando a pressão é liberada do punção superior. O material ressalta perto do centro superior do compacto, mas é momentaneamente restrito nas bordas devido ao arrasto de fricção entre o compacto e a parede da matriz. Isso resulta em uma tensão de tração concentrada na borda superior do compacto. Rachaduras devido a esse mecanismo (chamado de finalização) podem ser evitadas por (1) uso de um lubrificante para minimizar a fricção da matriz; (2) aumento da resistência verde da peça por meio da seleção de ligantes; (3) minimização do ressalto; e (4) manutenção de uma pressão de retenção no punção superior durante a ejeção.

O segundo mecanismo é ilustrado na Figura 13.13b. Isso também envolve ressalto. À medida que a peça limpa o topo da matriz durante a ejeção, o material ressalta para uma seção transversal maior. Isso cria uma tensão de tração no material logo acima do topo da matriz e pode resultar em uma série de rachaduras laminares. Esse mecanismo pode ser minimizado pela seleção de um sistema de ligantes que forneça boa resistência verde com um ressalto mínimo.

\includegraphics*[width=\linewidth]{./images/formacao_freatura_prensagem.png}

Uma terceira fonte de densidade verde não uniforme é a presença de aglomerados duros (aglomerados de partículas) no pó ou uma variedade de dureza dos grânulos em um pó granulado de fluxo livre. Os grânulos duros protegerão o pó ou grânulos mais macios circundantes da exposição à pressão máxima de prensagem, resultando em aglomerados de poros que reduzem a resistência. Às vezes, o pó circundante se compactará uniformemente, mas o aglomerado duro aprisionará porosidade. O aglomerado duro pode então encolher mais do que o material circundante durante a densificação e deixar um grande poro. Note que a causa da densidade não uniforme está mais associada à condição do pó carregado na matriz de prensagem do que com a própria operação de prensagem. Este problema pode não aparecer até a etapa de densificação na fabricação.

\subsection*{Prensagem isostática}

A prensagem uniaxial possui limitações conforme descrito na seção anterior. Algumas dessas limitações podem ser superadas aplicando pressão de todas as direções em vez de apenas uma ou duas direções. Isso é chamado de prensagem isostática ou prensagem isostática a frio. Também tem sido referido como prensagem hidrostática. A aplicação de pressão de múltiplas direções alcança uma maior uniformidade de compactação e aumento da capacidade de moldagem.

\includegraphics*[width=\linewidth]{./images/prensagem_isostatica.png}

Dois tipos de prensagem isostática são comumente utilizados: (1) prensagem com bolsa úmida e (2) prensagem com bolsa seca.

\paragraph*{Prensagem isostática com bolsa úmida}

O pó é selado em uma matriz à prova d'água. As paredes da matriz são flexíveis. A matriz selada é imersa em um líquido contido em uma câmara de alta pressão. A câmara é selada usando uma tampa com rosca ou de travamento. A pressão do líquido é aumentada por bombeamento hidráulico. As paredes da matriz se deformam e transmitem a pressão uniformemente para o pó, resultando em compactação. As paredes da matriz voltam ao estado original após a remoção da pressão, permitindo que o compacto seja removido facilmente da matriz depois que a tampa da matriz é removida.

Qualquer fluido não compressível pode ser usado para a prensagem isostática. A água é comumente utilizada, embora fluidos como óleo hidráulico e glicerina também funcionem. As paredes flexíveis da matriz ou molde são feitas de um elastômero, como borracha ou poliuretano. A flexibilidade e a espessura das paredes são cuidadosamente selecionadas para permitir um controle dimensional e características de liberação ótimas. Borracha natural, neoprene, borracha butílica, nitrilo, silicones, polissulfetos, poliuretanos e cloreto de polivinila plastificado já foram utilizados.

Uma preocupação importante na prensagem isostática é o preenchimento uniforme da matriz. Isso é geralmente alcançado pelo uso de vibração mais pó granulado ou secado por pulverização de fluxo livre. Uma vez que pressões mais elevadas são normalmente alcançadas pela prensagem isostática do que pela prensagem uniaxial e uma vez que essas pressões são aplicadas de forma uniforme, um maior grau de compactação é obtido. Isso geralmente resulta em características de densificação aprimoradas durante a etapa de sinterização subsequente do processamento e em um componente mais uniforme e livre de defeitos.

Assim como em outros processos, a prensagem isostática com bolsa úmida tem vantagens e desvantagens. As vantagens são a uniformidade de densidade, versatilidade e baixo custo de ferramentas. Associada à usinagem verde, uma ampla variedade e tamanho de peças podem ser fabricados com investimento mínimo em equipamentos. As desvantagens são o longo tempo de ciclo, alta exigência de mão de obra e dificuldade de automação. Os ciclos são de minutos e dezenas de minutos, portanto, as taxas de produção são baixas em comparação com a prensagem uniaxial.

\paragraph*{Prensagem isostatica com bolsa seca}

A prensagem isostática com bolsa seca foi desenvolvida para alcançar uma taxa de produção aumentada e tolerâncias dimensionais rigorosas. Em vez de imergir o ferramental em um fluido, o ferramental é construído com canais internos nos quais o fluido de alta pressão é bombeado. Isso minimiza a quantidade de fluido pressurizado necessária e permite o uso de ferramentas estacionárias. O desafio principal é construir o ferramental de forma que a pressão seja uniformemente transmitida ao pó para alcançar a forma desejada. Isso é realizado por meio de posicionamento e modelagem cuidadosos dos canais de fluido, muitas vezes pelo uso de vários materiais de elastômero diferentes em uma única matriz e pela otimização das restrições externas da matriz. Uma vez que um ferramental tenha sido projetado e automatizado corretamente, as peças podem ser prensadas a uma taxa de 1000-1500 ciclos por hora.

\subsection*{Aplicações da prensagem}

A prensagem uniaxial é amplamente utilizada para a compactação de formas pequenas, especialmente de cerâmicas isolantes, dielétricas e magnéticas para dispositivos elétricos. Estas incluem formas simples, como buchas, espaçadores, substratos e dielétricos de capacitores, e formas mais complexas, como as bases ou soquetes para tubos, interruptores e transistores. A prensagem uniaxial também é utilizada para a fabricação de telhas, tijolos, rodas de moagem, placas resistentes ao desgaste, cadinhos e uma variedade infinita de peças.

A prensagem isostática, normalmente em conjunto com a usinagem verde, é utilizada para configurações que não podem ser uniformemente prensadas uniaxialmente ou que exigem propriedades aprimoradas. Componentes grandes, como radomes, classificadores cônicos e envelopes de tubo de raios catódicos, foram fabricados por prensagem isostática. Componentes volumosos para a indústria de papel também foram produzidos. Pequenos componentes com uma grande relação comprimento-largura também são fabricados por prensagem isostática e usinagem.

\section*{Fundição/Moldagem (Casting)}

Uma quantidade significativa de fundição de cerâmica fundida é feita na preparação de refratários de $Al_2O_3$ de alta densidade e $Al_2O_3-ZrO_2-SiO_2$ e na preparação de alguns materiais abrasivos ou refratários agregados (grão). Ao fabricar abrasivos, a fundição a partir de uma fusão em placas metálicas resfriadas produz têmpera rápida, o que resulta em tamanho de cristal muito pequeno que confere alta resistência ao material. A técnica de fundição de refratários fundidos é chamada de fundição por fusão.

A fundição de cerâmica frequentemente é feita por uma operação em temperatura ambiente na qual partículas cerâmicas suspensas em um líquido são lançadas em um molde poroso que remove o líquido e deixa um compacto de partículas no molde. Existem várias variações desse processo, dependendo da viscosidade da suspensão cerâmica-líquida, do molde e dos procedimentos utilizados. A mais comum é referida como fundição de suspensão. Os princípios e controles para a fundição de suspensão são semelhantes aos das outras técnicas de fundição de cerâmica de partículas. A fundição de suspensão é descrita em detalhes, seguida por uma breve descrição de outras técnicas.

\subsection*{Fundição por suspensão / Moldagem por barbotina}

A maioria das fundições por suspensão comerciais envolve partículas cerâmicas suspensas em água e lançadas em moldes de gesso poroso.

\paragraph*{Materiais brutos}

A seleção do pó inicial depende dos requisitos das aplicações. A maioria das aplicações requer um pó fino, tipicamente de malha $-$325 (44 $\mu$m). Aplicações que exigem alta resistência requerem pós ainda mais finos, com média abaixo de 5 $\mu$m, com uma parte substancial abaixo de 1 $\mu$m. A composição química frequentemente é uma consideração importante na seleção do pó inicial e aditivos. Impurezas e fases secundárias podem ter efeitos pronunciados nas propriedades em altas temperaturas.

\paragraph*{Processamento de pó}

O processamento para a fundição em suspensão geralmente envolve a classificação de partículas para obter uma distribuição de tamanho de partícula que resultará em empacotamento máximo e uniformidade durante a fundição. Frequentemente, a classificação de partículas é combinada em uma etapa com a adição de ligantes, agentes umectantes, defloculantes e auxiliares de densificação, e com a preparação da suspensão. Isso geralmente é feito por moagem de esferas, mas também pode ser feito por moagem vibratória ou outros processos que proporcionem moagem úmida. Após a moagem, a suspensão é peneirada e talvez passada por um separador magnético para remover a contaminação por ferro. Pode ser necessário um ajuste leve para alcançar a viscosidade desejada e, em seguida, a suspensão está pronta para o envelhecimento, desaeração ou fundição.

\paragraph*{Preparação de suspensão e reologia}

\begin{definition}[Reologia]
    A reologia é o estudo das características de fluxo da matéria, por exemplo, suspensões de partículas sólidas em um líquido, e é descrita quantitativamente em termos de viscosidade $\eta$. Para baixas concentrações de partículas esféricas onde não ocorre interação entre as partículas, aplica-se a relação de Einstein:

    \begin{gather}
        \frac{v}{v_0} = 1 + 2,5V
    \end{gather}

    onde $v$ é a viscosidade da suspensão, $v_0$ é a viscosidade do fluido de suspensão e $V$ é a fração de volume das partículas sólidas. Essa relação idealizada implica que a viscosidade resultante é controlada pela fração de volume de sólidos.

\end{definition}

Em sistemas reais, a fração de volume tem um grande efeito, mas também o têm o tamanho das partículas, a forma das partículas, as cargas superficiais das partículas e o grau de aglomeração versus dispersão. Todos esses fatores estão inter-relacionados. A viscosidade é essencialmente determinada pela proximidade das partículas umas às outras e pelo grau de atração ou repulsão entre as partículas.

\paragraph*{Tamanho de partícula e efeitos de geometria}

A distribuição do tamanho das partículas influencia diretamente a viscosidade da suspensão, afetando sua capacidade de fluir e preencher uniformemente o molde durante o processo de fundição. Partículas menores tendem a resultar em uma viscosidade mais baixa, facilitando o fluxo e melhorando a capacidade de preenchimento de detalhes complexos no molde. Por outro lado, partículas maiores podem aumentar a viscosidade, dificultando o fluxo e afetando a uniformidade do preenchimento. Além disso, a forma geométrica das partículas pode influenciar a maneira como elas se empacotam na suspensão, afetando a densidade e a capacidade de empacotamento das partículas.

\paragraph*{Efeitos da superfície da partícula}

Para suspensões de alto teor de sólidos, a atração partícula-partícula resulta na formação de aglomerados. Em alguns casos, esses aglomerados podem agir essencialmente como partículas aproximadamente esféricas e resultar em uma diminuição na viscosidade. Em outros casos, especialmente para teores de sólidos muito altos, os aglomerados podem interagir entre si e aumentar a viscosidade. O grau de aglomeração pode ser controlado com aditivos.

A dispersão e a floculação (aglomeração) de partículas cerâmicas em um fluido são fortemente afetadas pelo potencial elétrico na superfície da partícula, íons adsorvidos e distribuição de íons no fluido adjacente à partícula. Assim, a estrutura química e eletrônica do sólido, o pH do fluido e a presença de impurezas são todas considerações críticas na preparação de uma suspensão para fundição.

Duas abordagens são comumente usadas para controlar e manipular as características de superfície de partículas cerâmicas em uma suspensão: (1) repulsão eletrostática e (2) estabilização estérica.

A repulsão eletrostática envolve a acumulação de cargas da mesma polaridade em todas as partículas. Cargas semelhantes se repelem, então as partículas são mantidas afastadas na suspensão por forças eletrostáticas. Quanto maior a carga elétrica na superfície das partículas, melhor o grau de dispersão e menos aglomeração. A carga na superfície das partículas é controlada pelo pH do líquido e pela adição de produtos químicos que fornecem cátions monovalentes ($Na^+$, $NH_4^+$, $Li^+$) para absorção na superfície das partículas.

Para a maioria dos óxidos, a dispersão pode ser controlada pelo pH usando as propriedades polares da água e as concentrações de íons de ácidos ou bases para alcançar zonas carregadas ao redor das partículas para que elas se repilam umas às outras. Materiais de argila também podem ser dispersos por repulsão eletrostática. A obtenção da dispersão ótima pode ser auxiliada pelo uso de vários equipamentos: um medidor de pH, um zetametro e um viscosímetro.

O controle da dispersão (defloculação) e aglomeração (floculação) pode ser alcançado com repulsão eletrostática. Uma segunda abordagem importante é chamada de estabilização estérica ou obstrução estérica. Ela envolve a adição de moléculas orgânicas em forma de corrente que são adsorvidas nas partículas cerâmicas e fornecem uma zona de buffer ao redor de cada partícula. Uma extremidade da corrente se fixa ou âncora à cerâmica e tem solubilidade limitada no solvente. A outra extremidade se estende para longe da partícula e é solúvel no solvente. Essas moléculas fornecem uma barreira física para a aglomeração.

Vários fatores influenciam a obstrução estérica:

\begin{itemize}
    \item A afinidade de uma extremidade de uma molécula em cadeia para ser adsorvida na superfície da partícula cerâmica.
    \item A resistência da extremidade da cauda da molécula para se fixar nas extremidades das caudas de moléculas adjacentes.
    \item As características do fluido, o comprimento da molécula orgânica.
\end{itemize}

As suspensões aquosas utilizando repulsão eletrostática são comumente usadas para a fundição de suspensões. Suspensões não aquosas que utilizam obstrução estérica são comumente usadas para a fundição em fita.

\paragraph*{Preparação da suspensão}

A preparação física real da suspensão pode ser feita por uma variedade de técnicas. Talvez a mais comum seja a moagem ou mistura úmida em moinho de bolas. Os ingredientes, incluindo o pó, ligantes, agentes umectantes, auxiliares de sinterização e agentes dispersantes, são adicionados ao moinho com a proporção adequada do líquido de fundição selecionado e moídos para alcançar uma mistura completa, umedecimento e, geralmente, redução do tamanho das partículas. A suspensão é então deixada para envelhecer até que suas características sejam relativamente constantes. Em seguida, está pronta para verificação final de viscosidade (e ajuste, se necessário), desaeração e fundição.

\paragraph*{Preparação do molde}

O molde para fundição de suspensão deve ter porosidade controlada para que possa remover o fluido da suspensão por ação capilar. O molde também deve ter um custo baixo. O material de molde tradicional tem sido gesso. Alguns moldes mais novos, especialmente os de fundição por pressão, são feitos de um material plástico poroso. É essa porosidade que retira a água da suspensão durante a fundição. A quantidade de porosidade pode ser controlada pela quantidade de água em excesso adicionada durante a fabricação do molde de gesso. Para fundição de suspensão normal, são usados 70 a 80\% em peso de água. A taxa de configuração do gesso pode ser amplamente variada por impurezas.

\paragraph*{Fundição}

Depois que o molde foi fabricado e secado adequadamente e uma suspensão ideal foi preparada, a fundição pode ser realizada. Muitas opções estão disponíveis, dependendo da complexidade do componente e de outros fatores:

\begin{itemize}
    \item Fundição simples em um molde de uma peça
    \item Fundição simples em um molde de várias peças
    \item Fundição de drenagem
    \item Fundição sólida
    \item Fundição a vácuo
    \item Fundição centrífuga
    \item Fundição sob pressão
    \item Fundição com molde solúvel
    \item Fundição em gel
    \item Fundição com pinos ou mandris não absorventes inseridos no molde
\end{itemize}

A fundição de drenagem, por outro lado, envolve o despejo da suspensão no molde, e a água é sugada onde a suspensão está em contato com o molde, deixando uma deposição de partículas densamente compactada crescendo na suspensão a partir das paredes do molde. A suspensão é deixada no molde até que a espessura desejada seja alcançada, momento em que o restante da suspensão é drenado do molde. A fundição de drenagem é a abordagem mais comum para a fundição de suspensão. É utilizada para a fundição de arte (estatuetas), pias e outros itens sanitários, cadinhos e uma variedade de outros produtos.

A fundição sólida é idêntica à fundição de drenagem, exceto que a suspensão é continuamente adicionada até que uma fundição sólida tenha sido alcançada.

A fundição a vácuo pode ser conduzida com a abordagem de drenagem ou sólida. Um vácuo é aplicado ao redor do lado de fora do molde. O molde pode consistir em uma forma permeável rígida ou em uma fina membrana permeável (como papel de filtro) revestindo uma forma rígida porosa. A fundição a vácuo é comumente utilizada na produção de placas de fibra refratária porosa para revestir fornos de alta temperatura.

A fundição centrífuga envolve a rotação do molde para aplicar cargas gravitacionais superiores ao normal, para garantir que a suspensão preencha completamente o molde. Isso pode ser benéfico na fundição de algumas formas complexas.

Uma limitação da maioria dos processos de fundição de suspensão é o tempo longo necessário para fundir os artigos no molde. Isso resulta em um grande estoque de moldes, mão de obra elevada e grande espaço no chão, o que aumenta os custos. A aplicação de pressão na suspensão aumenta a taxa de fundição. Isso é conhecido como fundição sob pressão. O uso de pressão na fundição foi inicialmente conduzido com moldes de gesso. No entanto, devido à baixa resistência do gesso, a quantidade de pressão que poderia ser aplicada era limitada. O desenvolvimento de moldes de plástico porosos permitiu que a pressão fosse aumentada em 10 vezes os valores de $3-4 MPa$ ($30-40 bar$ ou $435-580 psi$).

Peças de moldes não porosos e não absorventes, como pinos e mandris, podem ser usadas para alcançar maior complexidade de peças fundidas pelas técnicas mencionadas acima. A Figura 13.35 mostra esquematicamente como um mandril e pinos foram usados para fundir por drenagem um combustor de formato complexo para um motor a gás. A peça acabada é mostrada na Figura 13.36.

Maior complexidade de formato pode ser alcançada usando a técnica de fundição de moldes solúveis. Este método é baseado na tecnologia muito mais antiga da fundição por investimento. Também é conhecido como fundição de suspensão com cera fugitiva e é realizado nas seguintes etapas:

Um padrão de cera da configuração desejada é produzido por moldagem por injeção de uma cera solúvel em água.
O padrão de cera solúvel em água é mergulhado em uma cera não solúvel em água para formar uma camada fina sobre o padrão.
A cera de padrão é dissolvida em água, deixando a cera não solúvel em água como um molde preciso da forma.
O molde de cera é aparado, fixado a um bloco de gesso e preenchido com a suspensão de fundição apropriada.
Após a conclusão da fundição, o molde é removido dissolvendo-se em um solvente.
A forma fundida é seca, usinada conforme necessário e densificada em alta temperatura.

Uma técnica final de fundição é a deposição eletroforética. Ela utiliza uma carga eletrostática para consolidar partículas cerâmicas a partir de uma suspensão. Uma polaridade elétrica é aplicada ao molde que é oposta à polaridade na superfície das partículas cerâmicas. As partículas cerâmicas são atraídas eletricamente para a superfície do molde e se depositam como um compacto uniforme. Quando a espessura desejada do depósito é alcançada, ou o molde é removido do recipiente de suspensão, ou a suspensão é despejada do molde. A deposição eletroforética é geralmente utilizada para depositar um revestimento fino ou para produzir um corpo de parede fina, como um tubo. Também é usada para obter uma deposição muito uniforme de tinta spray em uma superfície condutora.

\paragraph*{Controle do processo de fundição}

O controle cuidadoso do processo é necessário no processo de fundição por deslizamento. Alguns dos fatores críticos incluem os seguintes:

\begin{itemize}
    \item Constância das propriedades
    \item Viscosidade
    \item Taxa de sedimentação
    \item Ausência de bolhas de ar
    \item Taxa de fundição
    \item Propriedades de drenagem
    \item Retração
    \item Propriedades de liberação
    \item Resistência
\end{itemize}

A constância das propriedades refere-se à reprodutibilidade do deslizamento de fundição e à sua estabilidade em função do tempo. O deslizamento deve ser facilmente reproduzido e preferencialmente não deve ser excessivamente sensível a pequenas variações no teor de sólidos e na composição química ou ao tempo de armazenamento. A viscosidade deve ser baixa o suficiente para permitir o preenchimento completo do molde, mas o teor de sólidos deve ser alto o suficiente para alcançar uma taxa de fundição razoável. Uma fundição muito lenta pode resultar em variações de espessura e densidade devido à sedimentação. Uma fundição muito rápida pode resultar em paredes cônicas (para fundição de drenagem), falta de controle de espessura ou bloqueio de passagens estreitas no molde.

O deslizamento deve estar livre de ar aprisionado ou reações químicas que produziriam bolhas de ar durante a fundição. Bolhas de ar presentes no deslizamento serão incorporadas na fundição e podem ser defeitos críticos na peça densificada final.

Uma vez que a fundição tenha sido concluída, a peça começa a secar e encolher longe do molde. Este encolhimento é necessário para conseguir liberar a peça do molde. Se a fundição grudar no molde, ela geralmente será danificada durante a remoção e rejeitada. A liberação do molde pode ser auxiliada revestindo as paredes do molde com um agente de liberação, como silicone ou óleo de oliva. No entanto, deve-se reconhecer que o revestimento pode alterar a taxa de fundição.

A resistência da fundição deve ser adequada para permitir a remoção do molde, a secagem e o manuseio antes da operação de queima. Às vezes, uma pequena quantidade ($<1\%$) de ligante é incluída no deslizamento. Ligantes orgânicos como PVA funcionam bem. Com o ligante presente, é possível alcançar uma resistência comparável ou superior à do giz de lousa. Essa resistência é adequada para manuseio e também para usinagem a verde, se necessário.

\paragraph*{Secagem}

Os compactos fabricados por fundição são saturados com o fluido de fundição. O fluido fica retido em todos os poros, além de formar um filme em algumas das partículas. O fluido deve ser completamente removido por uma operação de secagem antes que o compacto possa ser levado a altas temperaturas para densificação.

A facilidade de remoção do fluido depende de vários fatores: (1) a quantidade de porosidade; (2) o tamanho dos canais de poros interconectados; (3) a pressão de vapor do fluido; e (4) a espessura do compacto.

A distribuição da porosidade é particularmente importante. Poros grandes e canais de poros permitem a remoção fácil do fluido. No entanto, esses resultam em baixa densidade verde, contração durante a secagem e dificuldade de queima para uma microestrutura densa e de grão fino. Por outro lado, compactos fundidos com partículas compactas, poros pequenos e canais de poros estreitos são fáceis de densificar para uma microestrutura de grão fino, mas são difíceis de secar. Os canais finos resultam em grandes pressões capilares e tensões que podem facilmente rachar o compacto se o fluido não for removido de forma uniforme.

\subsection*{Fundição/moldagem por fita (Tape Casting)}

Algumas aplicações, como substratos e embalagens para eletrônicos e dielétricos para capacitores, requerem finas folhas de cerâmica. A fundição em fita foi desenvolvida para fabricar essas finas folhas em grande quantidade e a baixo custo. É semelhante à fundição de suspensão, exceto que a suspensão contém cerca de 50\% em volume de aglutinante orgânico e é espalhada sobre uma superfície plana em vez de ser despejada em um molde moldado.

\paragraph*{Processo doctor blade}

A abordagem mais comum para a fundição de fita é o processo de "doctor blade". A técnica consiste em lançar uma lama sobre uma superfície transportadora em movimento (geralmente um filme fino de acetato de celulose, Teflon, Mylar ou celofane) e espalhar a lama para uma espessura controlada com a borda de uma lâmina longa e lisa. A lama contém um sistema de aglutinantes dissolvidos em um solvente. É necessário que haja aglutinante suficiente para que uma fita flexível resulte quando o solvente for removido. A remoção do solvente é realizada por evaporação. Assim como na fundição de suspensão, o fluido deve ser removido lentamente para evitar rachaduras, bolhas ou distorções. Esse é o propósito da porção longa do aparelho de fundição de fita entre a lâmina do doutor e o carretel de recolhimento. A evaporação é realizada por aquecimento controlado ou por fluxo de ar. A fita seca e flexível é enrolada em um carretel para ser armazenada para uso.

\paragraph*{Outros processos de tape casting}

Um segundo processo de fundição de fita é a técnica da "cascata". A lama é bombeada em um sistema de recirculação para formar uma cortina contínua. Uma correia transportadora transporta uma superfície plana através da lama. A camada uniforme e fina de lama no transportador é então transferida por correia transportadora para a etapa de secagem. Esta técnica tem sido utilizada para formar uma fita fina para dielétricos de capacitores e uma fita mais espessa para eletrodos porosos para células de combustível. Também é comumente usada para aplicar o revestimento de chocolate em barras de chocolate. Um terceiro processo de fundição de fita é o processo de fundição de papel. Um papel com baixo teor de cinzas passa por uma lama. A lama umedece o papel e adere a ele. A espessura de aderência depende da viscosidade da lama e da natureza do papel. O papel revestido passa por um secador, e a fita resultante é enrolada em um carretel de recolhimento. O papel é posteriormente removido durante um processo de queima. Esta técnica tem sido utilizada na fabricação de estruturas tipo favo de mel para trocadores de calor.

\paragraph*{Preparação de lamas para moldagem em fita}

A preparação de lamas para moldagem em fita é similar do ponto de vista reológico às lamas para fundição, mas contém uma quantidade maior de ligante. Além disso, o sistema de ligante e plastificante é geralmente selecionado para ser termoplástico, ou seja, pode ser amolecido por aquecimento a temperaturas moderadas. Isso permite que as camadas sejam unidas por laminação.

As características dos componentes orgânicos comumente adicionados para obter uma lama de fundição de fita aceitável incluem um ligante, um plastificante, um dispersante, um agente umectante e um agente antiespumante. Cada combinação deve ser selecionada e otimizada para um fluido específico (solvente). Exemplos de solventes são MEK, álcoois, tolueno, hexano, tricloroetileno e água. Exemplos de ligantes são polivinil butiral, acetato de polivinila, cloreto de polivinila, PVA, emulsão acrílica, poliestireno, polimetacrilatos e nitrato de celulose.

Os critérios para o ligante incluem: (1) formação de um filme resistente e flexível quando seco; (2) volatilização para um gás quando aquecido e não deixa resíduo de carbono ou cinzas; (3) permanece estável durante o armazenamento, especialmente sem alteração no peso molecular; e (4) é solúvel em um solvente barato, volátil e não inflamável.

\paragraph*{Aplicações do Tape Casting}

As principais aplicações da fundição de fita são para a fabricação de dielétricos para capacitores cerâmicos multicamadas (MLCCs) e de Al2O3 para substratos e pacotes multicamadas para circuitos integrados.

% \section*{Conformação plástica}

