\lecture{1}{}{Ensaios mecânicos}

\begin{multicols*}{2}

\section*{Ductilidade em Materiais}

  \begin{itemize}
    \item Limite de Escoamento
    \item Limite de Resistência
    \item Resistência à fratura
    \item Região elástica - Definição? Deve ser linear? 
    \item Região plástica - Movimentos de discordância;
    \item Qual o comportamento (os perfis) das curvas tensão-deformação de engenharia?
    \item Fadiga e fragilidade do material
    \item Quais os tipos de fraturas em materiais dúcteis?
    \item Movimento de discordâncias - compactação - densidade menor nos cerâmicos (2 ou mais espécies químicas); Quais são os sistemas de deslizamento?; O que move as discordâncias são as tensões de cisalhamento; baixa simetria.
    \item Qual a condição necessária para haver comportamento dúctil num material?
    \item Quais materiais cerâmicos possuem comportamento dúctil?
  \end{itemize}

\section*{Fragilidade em materiais}

  \begin{itemize}
    \item Resistência à fratura
    \item Qual a condição necessária para que o material tenha comportamento frágil?
  \end{itemize}

\section*{Propriedades mecânicas de interesse}

  \begin{itemize}
    \item Módulo de elasticidade - E vs. Tipo de ligação química vs. Temperatura vs. Distância interplanar vs. Porosidade (densidade); depende da densidade; depende do processamento (feito em fase líquida? queda brusca de temperatura em relação ao E); distância interplanar 
    \item Resistência à fratura
    \item Uso de ultrassom (em contraste com o ensaio de tração e AFM) para medir o $E$
    \item O que é a tensão principal? (No plano perpendicular, a tensão de cisalhamento é igual a zero)
    \item Extensometria - strain gauge;
    \item AFM - capaz de calcular densidade de 100\% (referencial)
    \item Resistência teórica
    \item Teoria de Inglis
    \item Limite de escoamento - movimento de discordâncias em altíssimas temperaturas
    \item Razão entre ensaio de tração e compressão - qual a magnitude da tensão necessária para causar a mesma tensão local em relação aos dois ensaios?
    \item Ensaio de flexão - módulo de ruptura
    \item Volume efetivo
    \item A resistência mecânica de um material cerâmico é uma propriedade intrínseca? não
  \end{itemize}

\section*{Ensaio de resistência mecânica}

  \begin{itemize}
    \item O tipo de ensaio utilizado depende do estado de tensão que o material cerâmico será submetido
    \item Teoria estatística de Weibull - teoria do ligamento mais fraco 
    \item Módulo de Weibull - é utilizado para calibrar a dispersão do material no cálculo da probabilidade à falha - depende do desenvolvimento microestrutural (não só de densificação e processamento)
    \item A quantidade de corpos de prova que devem ser testados depende da função estimadora utilizada no ensaio
    \item Fratografia
    \item Volume Efetivo
    \item Desalinhamento do sistema
  \end{itemize}

\end{multicols*}
