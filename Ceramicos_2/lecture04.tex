\lecture{0.4}{}{Sinterização}

\section*{Introdução}

Até o momento, a microestrutura tem sido subestimada em sua influência sobre as propriedades cerâmicas, em relação ao módulo de Young, a expansão térmica, a condutividade elétrica, os pontos de fusão e a densidade. No entanto, ao longo deste estudo, torna-se evidente que a microestrutura desempenha um papel significativo na determinação das propriedades dos materiais. Em contraste com metais e polímeros, a produção de cerâmicas apresenta desafios devido à alta refratariedade e à natureza quebradiça desses materiais. Em geral, as cerâmicas são processadas a partir de pós finos que são moldados em formas desejadas e em seguida sinterizados para alcançar uma densidade sólida.

A sinterização, crucial para o processo de fabricação cerâmica, envolve a conversão de um compacto de pó em um corpo cerâmico denso e resistente por meio do aquecimento controlado.

\begin{remark}
    A sinterização é entendida como quaisquer mudanças de forma que uma pequena partícula ou um aglomerado de partículas de composição uniforme sofre quando mantido a alta temperatura.
\end{remark}

\begin{definition}[Sinterização (definição de engenharia)]
    A sinterização é um processo de consolidação através de difusão com consequente crescimento de grão e diminuição da porosidade, resultando em aumento de densidade. A consolidação se dá pela ligação entre as partículas originalmente em contato ou soltas.
\end{definition}

Embora a sinterização possa ocorrer na presença ou ausência de uma fase líquida, a sinterização em fase líquida é preferida tecnologicamente, pois permite o controle mais eficaz da densificação do material. No entanto, alcançar a densidade teórica durante a sinterização é um desafio devido à pequena força motriz associada ao processo, tornando a obtenção de densidade total uma tarefa complexa que requer cuidadoso controle dos parâmetros de processamento.

Certos aditivos são frequentemente incorporados ao pó cerâmico durante o processo de sinterização para controlar a taxa de densificação e minimizar a formação de porosidade. Esses aditivos podem incluir compostos de sinterização, como óxidos de metais alcalinos e alcalino-terrosos, que atuam como líquidos durante a sinterização, preenchendo espaços vazios e facilitando a movimentação das partículas. Além disso, agentes de sinterização como o bórax e o dióxido de titânio podem ser usados para promover a difusão atômica nas interfaces de partículas, facilitando a ligação entre as partículas individuais.

Durante o processo de sinterização, a taxa de aquecimento e resfriamento deve ser cuidadosamente controlada para evitar tensões térmicas excessivas que possam levar à formação de trincas ou falhas estruturais. O controle da atmosfera de sinterização também desempenha um papel crucial, pois a presença de gases reativos pode afetar significativamente as reações químicas entre os constituintes cerâmicos, alterando a microestrutura e as propriedades finais do material.

Além disso, o tamanho das partículas do pó cerâmico inicial, a distribuição do tamanho das partículas e a natureza dos aglomerados presentes também influenciam diretamente a cinética e a eficiência do processo de sinterização. Partículas finas tendem a se fundir mais rapidamente durante a sinterização, resultando em uma maior densificação, enquanto a presença de aglomerados pode levar a uma distribuição de porosidade não uniforme no material sinterizado final. Portanto, a seleção cuidadosa dos parâmetros de processamento e a compreensão detalhada dos mecanismos de sinterização são essenciais para a obtenção de propriedades cerâmicas desejáveis e consistentes.

\section*{Sinterização no estado sólido}

A força motriz macroscópica atuante durante a sinterização é a redução da energia excessiva associada às superfícies. Isso pode ocorrer por (1) redução da área superficial total através do aumento do tamanho médio das partículas, levando ao crescimento, e/ou (2) eliminação das interfaces sólido/vapor e criação de área de fronteira de grão, seguida pelo crescimento de grãos, o que leva à densificação.

Esses dois mecanismos geralmente competem entre si. Se os processos atômicos que levam à densificação predominam, os poros diminuem de tamanho e desaparecem com o tempo, e o compacto encolhe. Mas se os processos atômicos que levam ao crescimento forem mais rápidos, tanto os poros quanto os grãos aumentam de tamanho com o tempo.

Uma condição necessária para a densificação ocorrer é que a energia do contorno de grão $\gamma_{gb}$ seja ao menos duas vezes menor que a energia de superfície sólido vapor, $\gamma_{sv}$. Portanto, o ângulo de equilíbrio diedral, definido como:

\begin{gather}
    \gamma_{gb} = 2 \gamma_{sv} \cos(\frac{\phi}{2})
\end{gather}

deve ser menor que 180º. Para vários sistemas óxidos, o ângulo diedral é por volta de 120º, implicando que a relação $\frac{\gamma_{gb}}{\gamma_{sv}} \approx 1.0$, em contraste com sistemas metálicos, onde essa razão é entre 0.25 e 0.5.

\subsection*{Cinética de sinterização}

Para entender o que está ocorrendo durante a sinterização, é necessário medir o encolhimento, o tamanho de grão e de poro em função das variáveis de sinterização, como tempo, temperatura e tamanho de partícula inicial. Se um compacto de pó encolhe, sua densidade aumentará com o tempo. Portanto, a densificação é melhor acompanhada medindo a densidade do compacto (quase sempre dada como uma porcentagem da densidade teórica) em função do tempo de sinterização. Isso é geralmente realizado dilatometricamente, onde o comprimento de um compacto de pó é medido em função do tempo em uma dada temperatura. Curvas de encolhimento típicas são dadas num gráfico para duas temperaturas diferentes T2 > T1. A taxa de densificação é uma função forte da temperatura.

Em contraste, se um compacto de pó se torna mais grosseiro, nenhum encolhimento é esperado em um experimento dilatométrico. Nesse caso, a cinética de crescimento é melhor acompanhada medindo-se o tamanho médio de partícula em função do tempo por meio de microscopia ótica ou eletrônica de varredura.

É útil traçar o comportamento resultante em trajetórias de tamanho de grão versus densidade. Tipicamente, um material seguirá o caminho onde tanto a densificação quanto o crescimento ocorrem simultaneamente. No entanto, para obter densidades quase teóricas, o crescimento deve ser suprimido até que a maior parte do encolhimento tenha ocorrido. Um pó que segue a trajetória de maximizar o crescimento de grão, no entanto, está destinado a permanecer poroso — a energia livre foi gasta, grãos grandes se formaram, mas, mais importante, poros grandes também. Uma vez formados, esses poros são cineticamente muito difíceis de remover e, como discutido abaixo, podem até ser termodinamicamente estáveis, caso em que seria impossível removê-los.

Um método alternativo de apresentar os dados de sinterização é onde a evolução temporal dos tamanhos de grão e de poro é plotada; o crescimento grosseiro leva a um aumento em ambos, enquanto a densificação elimina os poros.

\includegraphics*[width=\linewidth]{./images/tamanho_grao_densidade.png}

Observe que à medida que o tempo progride, o tamanho médio do grão aumenta, enquanto o tamanho médio do poro diminui. A densidade completa é obtida apenas quando os processos atômicos associados ao crescimento são suprimidos, enquanto aqueles associados à densificação são aprimorados. É imperativo entender o efeito da curvatura no potencial químico dos íons ou átomos em um sólido para poder controlar os processos que ocorrem durante a sinterização.

\includegraphics*[width=\linewidth]{./images/trajetoria_grao_densidade.png}

\paragraph*{Força motriz local para sinterização}

Conforme mencionado anteriormente, a força global que atua durante a sinterização é a redução da energia superficial, que se manifesta localmente como diferenças de curvatura. A partir da equação de Gibbs-Thompson, é possível observar que a diferença de potencial química por fórmula unitária $\Delta\mu$ entre átomos numa superfície plana e numa superfície de curvatura $\kappa$, e com volume de fórmula unitária $\Omega_{MX}$ é:

\begin{gather}
    \Delta\mu = \mu_{\text{curv}} - \mu_{\text{flat}} = \gamma_{sv} \Omega_{MX} \kappa
\end{gather}

A curvatura $\kappa$ depende da geometria (e.g. uma esfera de raio $\rho$, $\kappa = \frac{2}{\rho}$) A equação acima tem duas ramificações muito importantes que são críticas para a compreensão do processo de sinterização. A primeira está relacionada à pressão parcial de um material acima de uma superfície curva, e a segunda envolve o efeito da curvatura na concentração de vacâncias.

\paragraph*{Efeito da curvatura na pressão parcial}

No equilíbrio, a diferença de potencial químico é traduzida em uma diferença de pressões parciais acima da superfície curvada:

\begin{gather}
    \Delta\mu = kT ln \frac{P_{\text{curv}}}{P_{\text{flat}}} \\
    ln \frac{P_{\text{curv}}}{P_{\text{flat}}} = \kappa \frac{\gamma_{sv} \Omega_{MX}}{kT}
\end{gather}

Para uma esfera de raio $\rho$, onde $\kappa = \frac{2}{\rho}$, temos que:

\begin{gather}
    P_{\text{curv}} = P_{\text{flat}} \left( 1 + \frac{2\gamma_{sv} \Omega_{MX}}{\rho k T} \right)
\end{gather}

Dado que o raio de curvatura é definido como negativo para uma superfície côncava e positivo para uma superfície convexa, essa expressão é de fundamental importância, pois prevê que a pressão de um material acima de uma superfície convexa é maior do que aquela sobre uma superfície plana, e vice-versa para uma superfície côncava. Por exemplo, a pressão dentro de um poro de raio $\rho$ seria menor do que aquela sobre uma superfície plana; inversamente, a pressão ao redor de uma coleção de partículas esféricas finas será maior do que aquela sobre uma superfície plana. É somente ao apreciar esse fato que a sinterização pode ser compreendida.

Dada a importância dessa conclusão, devemos explorar o que ocorre no nível atômico que permite que isso aconteça. Considere o seguinte experimento mental: coloque cada uma das três superfícies de formato diferente do mesmo sólido em uma câmara selada e evacuada, e aqueça até que uma pressão de vapor de equilíbrio seja estabelecida. A análise das figuras mostra que $P1 < P2 < P3$, pois, em média, os átomos em uma superfície convexa estão menos firmemente ligados aos seus vizinhos do que os átomos em uma superfície côncava e, portanto, têm maior probabilidade de escapar para a fase gasosa, resultando em uma pressão parcial mais alta.

\includegraphics*[width=\linewidth]{./images/efeito_superficie_pressao_equilibrio.png}

\paragraph*{Efeito da curvatura na concentração de vacâncias}

Outra ramificação da diferença de potencial químico entre átomos de diferentes superfícies é que o equilibrio da concentração de vacâncias também é uma função da curvatura. A relação entre a concentração de equilibrio de vacâncias $C_0$, sua entalpia de formação $Q$ e a temperatura é dada por:

\begin{gather}
    C_0 = K \exp \left( - \frac{Q}{kT} \right)
\end{gather}

A entropia de formação e todos os termos pré-exponenciais estão inclusos na constante $K$. Foi assumido que as vacâncias foram formadas numa superfície plana sem estresse. Como o potencial químico de um átomo numa superfície curva é maior ou menor que numa superfície plana (dado por $\Delta\mu$), essa energia deve ser levada em consideração na equação acima, tendo portanto:

\begin{gather}
    C_{\text{curv}} = K \exp \left( - \frac{Q + \Delta\mu}{kT} \right) = C_0 \exp \left( - \frac{\gamma_{sv} \Omega_{MX} \kappa}{ k T} \right)
\end{gather}

E como na maior parte, $\gamma_{sv} \Omega_{MX} \kappa << kT$, temos:

\begin{gather}
    C_{\text{curv}} = C_0 \left( 1 - \frac{\gamma_{sv} \Omega_{MX} \kappa}{ k T} \right) \\
    \Delta C_{\text{vac}} = C_{\text{curv}} - C_0 = -C_0 \frac{\gamma_{sv} \Omega_{MX} \kappa}{ k T}
\end{gather}

Portanto, temos que a concentração de vacâncias numa superfície côncava é maior que numa superfície plana, que por sua vez, é maior do que numa superfície convexa.

Como conclusão: a curvatura causa variações locais nas pressões parciais e concentrações de vacâncias. A pressão parcial sobre uma superfície convexa é maior do que aquela sobre uma superfície côncava. Inversamente, a concentração de vacâncias sob uma superfície côncava é maior do que aquela abaixo de uma superfície convexa. Em ambos os casos, uma força motriz está presente, induzindo os átomos a migrarem das áreas convexas para as côncavas, ou seja, dos picos das montanhas para os vales. Com base nessas conclusões, é possível explorar os diversos mecanismos atômicos que ocorrem durante a sinterização.

\subsection*{Mecanismos atômicos que ocorrem durante a sinterização}

Basicamente, existem cinco mecanismos atômicos pelos quais a massa pode ser transferida em um compacto de pó:

\begin{itemize}
    \item Evaporação-condensação, representada pelo caminho 1.
    \item Difusão superficial, ou caminho 2.
    \item Difusão de volume. Aqui existem dois caminhos. A massa pode ser transferida da superfície para a área do pescoço - caminho 3, ou da área de fronteira de grão para a área do pescoço - caminho 5.
    \item Difusão de fronteira de grão da área de fronteira de grão para a área do pescoço - caminho 4.
    \item Fluxo viscoso ou fluência. Esse mecanismo implica a deformação plástica ou o fluxo viscoso de partículas de áreas de alto estresse para baixo estresse e pode levar à densificação.
\end{itemize}

\includegraphics*[width=\linewidth]{./images/mecanismos_atomicos_coalescimento_densificacao.png}

\paragraph*{Coalescimento}

É importante notar que qualquer mecanismo no qual a fonte de material seja a superfície das partículas e o destino seja a área do pescoço não pode levar à densificação, pois tal mecanismo não permite que os centros das partículas se aproximem. Consequentemente, evaporação-condensação, difusão superficial e difusão de rede da superfície para a área do pescoço não podem levar à densificação. No entanto, esses processos resultam em uma mudança na forma dos poros, um aumento no tamanho do pescoço e um aumento concomitante na resistência do compacto. Além disso, os grãos menores, com seus menores raios de curvatura, tendem a "evaporar" e se depositar nas partículas maiores, resultando em um aumento no tamanho médio dos grãos.

A força motriz em todos os casos é a diferença de pressão parcial associada às variações locais de curvatura. Por exemplo, a pressão parcial no ponto $s$ na figura acima é maior do que a no ponto $n$, o que resulta na transferência de massa das superfícies convexas para as côncavas. O caminho real a ser percorrido dependerá da cinética dos vários caminhos. Uma vez que os processos atômicos ocorrem em paralelo, em uma dada temperatura, é o mecanismo mais rápido que irá dominar.

\paragraph*{Densificação}

Se a transferência de massa da superfície para a área do pescoço ou da superfície de grãos menores para grãos maiores não leva à densificação, outros mecanismos devem ser invocados para explicar este último. Para que ocorra a densificação, a fonte de material deve ser o contorno de grão ou a região entre as partículas de pó, e o destino deve ser a região do pescoço ou do poro. Consequentemente, os únicos mecanismos, além da deformação viscosa ou plástica, que podem levar à densificação são a difusão de contorno de grão e a difusão em massa da área de contorno de grão para a área do pescoço. Atomisticamente, ambos os mecanismos envolvem a difusão de íons da região de contorno de grão em direção à área do pescoço, para a qual a força motriz é a concentração de vacâncias induzida pela curvatura. Como há mais vacâncias na área do pescoço do que na região entre os grãos, um fluxo de vacâncias se desenvolve a partir da superfície do poro em direção à área do contorno de grão, onde as vacâncias são eventualmente aniquiladas. Um fluxo atômico igual difundirá na direção oposta, preenchendo os poros.

\section*{Cinética de Sinterização}

Com base na discussão anterior, um compacto de pó pode reduzir sua energia seguindo vários caminhos, alguns dos quais podem levar ao engrossamento, outros à densificação. Isso traz à tona a questão central e crítica na sinterização: o que determina se uma coleção de partículas se densificará ou se engrossará? Por exemplo, um compacto em que a difusão superficial é muito mais rápida do que a difusão em massa tenderia a engrossar em vez de se densificar.

A cinética da sinterização depende de muitas variáveis, incluindo o tamanho e o empacotamento das partículas, a atmosfera de sinterização, o grau de aglomeração, a temperatura e a presença de impurezas. Devido à geometria complexa do problema, soluções analíticas são possíveis apenas ao fazer consideráveis aproximações geométricas e de campo de fluxo de difusão, o que raramente é realizado na prática. Consequentemente, os modelos discutidos a seguir têm validade limitada e devem ser usados com extremo cuidado ao tentar prever o comportamento de sinterização de pós reais. Grande parte da utilidade desses modelos de sinterização reside mais na compreensão das tendências gerais que se espera e na identificação dos parâmetros críticos do que em suas capacidades preditivas.

\subsection*{Estágios de sinterização}

Um estágio de sinterização é um "intervalo de mudança geométrica em que a forma do poro é totalmente definida (como o arredondamento dos pescoços durante o estágio inicial de sinterização) ou um intervalo de tempo durante o qual o poro permanece constante em forma, mas diminui em tamanho." Com base nessa definição, foram identificados três estágios: um inicial, um intermediário e um final. Durante o estágio inicial, a área de contato entre partículas aumenta por crescimento de pescoço, e a densidade relativa aumenta de cerca de 60 para 65 por cento.

O estágio intermediário é caracterizado por canais de poros contínuos coincidentes com três bordas de grãos. Durante esse estágio, a densidade relativa aumenta de 65 para cerca de 90 por cento, tendo a matéria difundindo-se em direção aos canais cilíndricos longos e os vazios afastando-se deles.

O estágio final começa quando a fase do poro é eventualmente fechada e é caracterizado pela ausência de um canal de poro contínuo. Os poros individuais têm forma lenticular, se estiverem nas fronteiras de grãos, ou arredondada, se estiverem dentro de um grão. Uma característica importante deste estágio é o aumento da mobilidade de poros e fronteiras de grãos, que precisam ser controlados se a densidade teórica for ser alcançada.

\begin{center}
    \includegraphics*[scale=0.45]{./images/estagios_sinterizacao.png}
\end{center}

Claramente, a cinética de sinterização será diferente durante cada um dos estágios mencionados anteriormente. Para complicar ainda mais as coisas, além de ter que tratar cada estágio separadamente, a cinética dependerá dos mecanismos atômicos específicos em operação. Apesar dessas complicações, a maioria, se não todos, os modelos de sinterização compartilham a seguinte filosofia comum:

\begin{itemize}
    \item Uma forma de partícula representativa é assumida.
    \item A curvatura da superfície é calculada como uma função da geometria.
    \item Uma equação de fluxo que depende do passo limitante da taxa é adotada.
    \item A equação de fluxo é integrada para prever a taxa de mudança geométrica.
\end{itemize}

\paragraph*{Estágio inicial de sinterização} Devido aos vários caminhos possíveis para um pó nesse estágio, os diversos mecanismos devem ser definidos e avaliados. Os mecanismos são:

\begin{itemize}
    \item Evaporação-condensação: A sua equação prevê que a taxa de crescimento da região do pescoço (1) é inicialmente bastante rápida, mas depois se estabiliza, (2) é uma função forte do tamanho inicial da partícula e (3) é uma função da pressão parcial $P_{\text{flat}}$ do composto, que por sua vez depende exponencialmente da temperatura.
    \item Difusão na rede
    \item Difusão pelo contorno de grão
    \item Difusão na superfície
    \item Sinterização viscosa
\end{itemize}

Normalmente, as energias de ativação para a difusão superficial, de contorno de grão e de rede aumentam nessa ordem. Assim, a difusão superficial é favorecida em temperaturas mais baixas e a difusão de rede em temperaturas mais altas. Comparando as equações de cada modelo, observa-se que a difusão de contorno de grão e de superfície são preferidas em relação à difusão de rede para partículas menores. A difusão de rede, no entanto, é favorecida em tempos de sinterização longos, altas temperaturas de sinterização e partículas maiores. No entanto, de longe, o mecanismo mais tolerante em relação ao tamanho da partícula é a sinterização viscosa. É importante observar que essas tendências gerais também se estendem às fases de sinterização intermediária e final.

\subsection*{Cinética de densificação}

\paragraph*{Estágio intermediário de sinterização}

A maior parte da densificação de um compacto de pó ocorre durante a fase intermediária. Infelizmente, essa fase é a mais difícil de abordar porque depende fortemente dos detalhes do empacotamento de partículas - uma variável que é bastante difícil de modelar. Para tornar o problema viável, fazemos as seguintes suposições:

\begin{itemize}
    \item O compacto de pó é composto por tetracaidecaedros empacotados de maneira ideal de comprimento $a_p$, separados uns dos outros por longos canais porosos de raios $r_c$.
    \item A densificação ocorre pela difusão em massa de vacâncias longe dos canais porosos cilíndricos em direção aos contornos de grão.
    \item Um perfil linear e em estado estacionário da concentração de vacâncias é estabelecido entre a fonte e o sumidouro.
    \item As vacâncias são aniquiladas nos contornos de grão; ou seja, os contornos de grão atuam como sumidouros de vacâncias. Também se assume que, onde as vacâncias são aniquiladas, sua concentração é dada por C0, ou seja, sob uma interface plana livre de estresse.
\end{itemize}

Fazendo essas suposições, pode-se mostrar que durante a sinterização da fase intermediária, a porosidade fracionária $P_c$ deve diminuir linearmente com o tempo.

\includegraphics*[width=\linewidth]{./images/estagio_intermediario_densificacao.png}

É bastante infeliz que, de todas as fases de sinterização, a mais importante seja também a mais difícil de modelar. Por exemplo, qualquer modelo de fase intermediária que não leve em conta os detalhes do empacotamento de partículas tem validade muito limitada. O que é interessante sobre esse processo é que ele é autoacelerador, uma vez que, à medida que o cilindro diminui de diâmetro, sua curvatura aumenta e o gradiente de concentração de vacâncias também aumenta. Esse processo não pode e não continua indefinidamente; à medida que os poros cilíndricos ficam mais longos e mais finos, em algum ponto eles se tornam instáveis e se desintegram em poros esféricos menores ao longo do contorno de grão e/ou nos pontos triplos entre os grãos. É nesse ponto que a fase intermediária de sinterização cede lugar à sinterização da fase final, onde tanto a aniquilação dos últimos remanescentes de porosidade quanto o simultâneo engrossamento, ou seja, crescimento de grãos, da microestrutura ocorrem.

\paragraph*{Eliminação de poros}

Quando os átomos difundem em direção aos poros e as vacâncias são transportadas para longe dos poros para um local de escoamento, como contornos de grãos, discordâncias ou superfícies externas do cristal, os poros serão eliminados.

\paragraph*{Efeito do ângulo diedral na eliminação de poros}

Até agora, deve estar claro que a forma dos poros e a fração volumétrica evoluem continuamente durante a sinterização, e compreender essa evolução é fundamental para entender como altas densidades teóricas podem ser alcançadas. Uma suposição implícita e fundamental feita na análise anterior é a existência de uma força motriz para encolher os poros o tempo todo - uma suposição nem sempre válida. Como discutido abaixo, sob algumas condições, os poros podem ser termodinamicamente estáveis.

Utilizando a equação de equilíbrio de ângulo diedral, podemos observar que quando os grãos ao redor de um poro se encontram de forma que $\displaystyle \gamma_{gb} = 2\gamma_{sv} \cos(\frac{\phi}{2})$, a força motriz para a migração dos contornos de grão e a redução dos poros é igual a zero. Portanto, se os poros serão completamente eliminados, eles devem possuir um número de coordenação menor que um valor crítico $n_c$.

Com base nesses resultados, pode-se concluir que aumentar o ângulo diedral deve, em princípio, ajudar nas últimas etapas da sinterização. No entanto, a situação não é tão simples, uma vez que também pode ser mostrado que a fixação de poros a limites é mais forte para valores de $\phi$ menores. Dado que, para eliminar poros, eles precisam permanecer fixados ao limite de grão, essa última propriedade sugere que ângulos diedros baixos ajudariam a evitar a separação de poros do limite e, assim, seriam benéficos. Por fim, observe que a quebra de canais no final da sinterização da etapa intermediária ocorre em menores frações volumétricas de poros à medida que o ângulo diedro diminui, o que novamente é benéfico.

\includegraphics*[width=\linewidth]{./images/eliminacao_poros_diedro.png}

\subsection*{Cinética de coalescimento e crescimento de grão}

Qualquer coleção de partículas coalescerá com o tempo, conforme permitido pela cinética, onde o coalescimento implica um aumento no tamanho médio das partículas do conjunto com o tempo. Os modelos de coalescimento preveem que a cinética de coalescimento é aprimorada para sólidos com altas energias de interface vapor/superfície e altas pressões de vapor.

Como observado anteriormente, durante as etapas finais da sinterização, além da eliminação de poros, ocorre um coalescimento geral da microestrutura por crescimento de grão. Durante esse processo, o tamanho médio do grão aumenta com o tempo, conforme os grãos menores são consumidos pelos maiores. Controlar e entender os processos que levam ao crescimento de grãos são importantes por duas razões. A primeira, discutida com mais detalhes nos capítulos subsequentes, está relacionada ao fato de que o tamanho do grão é um fator importante para determinar muitas das propriedades elétricas, magnéticas, ópticas e mecânicas de cerâmicas. A segunda está relacionada à supressão do que é conhecido como crescimento de grão anormal, que é o processo pelo qual um pequeno número de grãos cresce muito rapidamente para tamanhos que são mais de uma ordem de grandeza maiores que a média na população. Além do efeito prejudicial que os grandes grãos têm sobre as propriedades mecânicas, as paredes desses grandes grãos podem se afastar de porosidades, deixando-as presas dentro deles, o que, por sua vez, limita a possibilidade de obter densidades teóricas em tempos razoáveis.

É importante compreender a origem da força motriz responsável pelo coalescimento. Considere a microestrutura esquemática composta por grãos cilíndricos de curvaturas variadas. Uma vez que, nesta estrutura, o ângulo de equilíbrio diédrico deve ser 120°, segue-se que os grãos com mais de seis lados tendem a crescer, enquanto aqueles com menos de seis lados tendem a encolher. Isso ocorre por migração de fronteiras de grãos na direção das setas. Para compreender a origem da força motriz, considere o esquema em escala atômica de tal fronteira na figura abaixo. Nesse nível, deveria ser óbvio por que um átomo no lado convexo da fronteira preferiria estar no lado côncavo - ele teria, em média, uma ligação mais forte, ou seja, teria menor energia potencial. Consequentemente, os átomos saltarão da direita para a esquerda, o que significa que a fronteira do grão se moverá da esquerda para a direita, como mostrado na Figura 10.19a. Ao analisar o problema nesse nível, é fácil ver por que as fronteiras de grãos retas (ou seja, sem curvatura) seriam estáveis e não se moveriam.

\includegraphics*[width=\linewidth]{./images/equilibrio_graos_lados.png}

O processo acima é referido às vezes como Ostwald ripening, caracterizado por uma dependência quadrática do tamanho de grão com o tempo. Ostwald ripening é um fenômeno de crescimento de grãos no qual grãos menores dissolvem-se e redissolvem-se em grãos maiores nas proximidades. Esse processo é governado pela diferença na pressão de curvatura dos grãos. Grãos menores têm uma curvatura maior e, portanto, uma pressão química mais alta. Isso leva à difusão do material a partir dos grãos menores para os maiores, onde o material se deposita, levando ao crescimento contínuo dos grãos maiores em detrimento dos menores.

Durante a sinterização de materiais cerâmicos, a difusão de átomos acontece principalmente nas regiões de interface entre os grãos, e o processo de Ostwald ripening contribui para a eliminação dos grãos mais pequenos em favor dos grãos maiores e mais estáveis. Isso resulta em um crescimento geral dos grãos e em uma redução na energia livre do sistema.

\paragraph*{Efeito da microestrutura e química do contorno de grão na mobilidade do contorno}

Ao derivar a equação do fenômeno de Ostwald ripening, fez-se a suposição implícita de que os contornos grão estavam livres de poros, inclusões e essencialmente de solutos - uma ocorrência muito rara, de fato. Assim, a equação prevê as chamadas cinéticas intrínsecas de crescimento de grão. Desnecessário dizer que a presença de "segundas fases" ou solutos nas fronteiras pode ter um efeito dramático em sua mobilidade e, do ponto de vista prático, geralmente é a mobilidade dessas fases que limita a taxa. Para ilustrar a complexidade do problema, considere apenas alguns processos possíveis que limitam a taxa:

\begin{itemize}
    \item Mobilidade intrínseca da fronteira de grão discutida acima.
    \item Arrasto extrínseco ou de soluto. Se a difusão do soluto segregado nas fronteiras de grão for mais lenta do que a mobilidade intrínseca da fronteira de grão, isso se torna limitante da taxa. Em outras palavras, se a fronteira de grão em movimento deve arrastar o soluto junto, isso tende a desacelerar.
    \item A presença de inclusões (basicamente segundas fases) nas fronteiras de grão. Pode ser demonstrado que inclusões maiores têm mobilidades mais baixas do que as menores e que quanto maior a fração volumétrica de uma determinada inclusão, maior a resistência à migração da fronteira.
    \item Transferência de material através de uma fase de fronteira contínua. Por exemplo, em $Si_3N_4$, o movimento da fronteira pode ocorrer apenas se o silício e o oxigênio difundirem através do filme fino e vítreo que geralmente existe entre os grãos.
    \item Em alguns casos, a redissolução das inclusões de segunda fase ancoradas na fronteira na matriz pode limitar a taxa.
\end{itemize}

Além desses, as seguintes interações entre poros e fronteiras de grão podem ocorrer:

\begin{itemize}
    \item O que é verdade para segundas fases também é verdade para poros. Os poros não podem aumentar a mobilidade da fronteira, apenas a deixam inalterada ou a reduzem. Durante as etapas finais da sinterização, à medida que os poros encolhem, a mobilidade das fronteiras aumentará (veja abaixo).
    \item Os poros nem sempre encolhem - também podem crescer à medida que se movem ao longo ou interceptam uma fronteira de grão em movimento.
    \item Os poros podem crescer pelo mecanismo de Ostwald ripening.
    \item Os poros podem crescer por evolução reativa de gases e distensão da amostra.
\end{itemize}

À medida que os grãos aumentam de tamanho e os poros diminuem em número, a mobilidade dos grãos aumenta da mesma forma. Em alguns casos, em uma combinação de tamanho de grão e densidade, a mobilidade das fronteiras de grão se torna grande o suficiente para que os poros não consigam mais acompanhá-las; as fronteiras simplesmente se movem rápido demais para que os poros as sigam e, consequentemente, desprendem-se.

Se a densidade teórica deve ser alcançada, é importante que a trajetória da fronteira de grão versus densidade não cruze essa região de separação. A importância de ter os poros próximos às fronteiras de grão é devido à a migração da fronteira para baixo varrer e eliminar todos os poros em seu caminho. Os poros que ficam presos nos grãos permanecerão lá porque as distâncias de difusão entre fontes e sumidouros se tornam muito grandes.

Existem essencialmente duas estratégias que podem ser empregadas para evitar a separação dos poros, a saber, reduzir a mobilidade da fronteira de grão e/ou aumentar a mobilidade do poro.

\paragraph*{Crescimento anormal de grão}

Em alguns sistemas, foi observado que um pequeno número de grãos na população crescem rapidamente para tamanhos muito grandes em relação ao tamanho médio da população. Esse fenômeno é conhecido como crescimento anormal de grão (CAG). O CAG deve ser evitado pela mesma razão do desprendimento dos poros das fronteiras dos grãos.

Embora não esteja totalmente claro o que causa o CAG, há evidências crescentes de que ele está mais provavelmente associado à formação de uma fase líquida ou filmes líquidos muito finos nas fronteiras de grão. Esses líquidos podem resultar de dopantes adicionados intencionalmente ou simplesmente de impurezas no pó inicial. Não há dúvida de que pequenas quantidades de líquido podem resultar em substancial coarsening da microestrutura.

\paragraph*{Fatores que afetam a sinterização no estado sólido}

Tipicamente, uma cerâmica sinterizada em estado sólido é um material opaco contendo alguma porosidade residual e grãos muito maiores do que os tamanhos de partícula iniciais. Com base na discussão e nos modelos apresentados, é útil resumir os fatores mais importantes que controlam a sinterização. Implícito nos argumentos a seguir está o desejo de atingir a densidade teórica.

\begin{itemize}
    \item Temperatura. Como a difusão é responsável pela sinterização, aumentar a temperatura aumentará muito a cinética de sinterização, pois a difusão é termicamente ativada. Como mencionado anteriormente, as energias de ativação para a difusão em massa geralmente são mais altas do que aquelas para a difusão de superfície e de contorno de grão. Portanto, aumentar a temperatura geralmente aprimora os mecanismos de difusão em massa que levam à densificação.
    \item Densidade verde. Geralmente, existe uma correlação entre a densidade verde (antes da sinterização) e a densidade final, uma vez que quanto maior a densidade verde, menos volume de poros deve ser eliminado.
    \item Uniformidade da microestrutura verde. Mais importante do que a densidade verde é a uniformidade da microestrutura verde e a falta de aglomerados. A importância da eliminação de aglomerados é discutida com mais detalhes abaixo.
    \item Atmosfera. O efeito da atmosfera pode ser fundamental para a densificação de um compacto de pó. Em alguns casos, a atmosfera pode aumentar a difusão de uma espécie que controla a taxa, por exemplo, influenciando a estrutura de defeitos. Em outros casos, a presença de um certo gás pode promover o coarsening, aumentando a pressão de vapor e suprimindo totalmente a densificação. Outra consideração importante é a solubilidade do gás no sólido. Como a pressão do gás dentro dos poros aumenta à medida que eles encolhem, é importante escolher um gás de atmosfera de sinterização que se dissolva facilmente no sólido.
    \item Impurezas. O papel das impurezas não pode ser superestimado. A chave para muitos produtos comerciais bem-sucedidos foi a identificação da pitada certa de pó mágico. O papel das impurezas foi extensivamente estudado e, até o momento, seu efeito pode ser resumido da seguinte forma:
    \begin{itemize}
        \item Auxílios de sinterização. São adicionados intencionalmente para formar uma fase líquida. Também é importante notar que o papel das impurezas nem sempre é apreciado. A presença de impurezas pode formar eutéticos de baixa temperatura e resultar em cinética de sinterização aprimorada, mesmo em concentrações muito pequenas.
        \item Suprimir o coarsening, reduzindo a taxa de evaporação e diminuindo a difusão superficial. Um exemplo clássico é a adição de boro ao $SiC$, sem o qual o $SiC$ não se densificará.
        \item Suprimir o crescimento de grãos e diminuir a mobilidade do contorno de grãos.
        \item Aumentar a taxa de difusão. Uma vez que o íon que limita a taxa durante a sinterização é identificado, a adição do dopante apropriado que se dissolverá e criará lacunas nesse sub-rede deve, em princípio, aprimorar a cinética de densificação.
    \end{itemize}
    \item Distribuição de tamanho. Distribuições estreitas de tamanho de grão diminuirão a propensão ao crescimento anormal de grãos.
    \item Tamanho das partículas. Uma vez que a força motriz para a densificação é a redução na área superficial, quanto maior a área superficial inicial, maior a força motriz. Assim, pareceria que se deve usar o tamanho de partícula inicial mais fino possível e, embora, em princípio, este seja um bom conselho, na prática, partículas muito finas apresentam sérios problemas. À medida que a relação superfície/volume das partículas aumenta, as forças eletrostáticas e outras forças superficiais se tornam dominantes, o que leva à aglomeração. Ao aquecer, os aglomerados têm uma tendência a se fundirem em partículas maiores, o que não apenas dissipa a força motriz para a densificação, mas também cria grandes poros entre os aglomerados parcialmente sinterizados, que posteriormente são difíceis de eliminar.
\end{itemize}

A solução reside em trabalhar com a natureza em vez de contra ela. Em outras palavras, utilize as forças superficiais para deflocular coloidalmente os pós e mantê-los de aglomerar. No entanto, uma vez dispersos, os pós não devem ser secos, mas encaminhados diretamente para um molde ou dispositivo que lhes dê a forma desejada. O motivo é simples. Em muitos casos, a secagem reintroduz os aglomerados e frustra o propósito do processamento coloidal.

Para evitar encolhimento excessivo durante a remoção do fluido, são necessárias lamas fluíveis com uma alta fração de volume de partículas. Uma vez moldada, as propriedades reológicas da lama devem ser dramaticamente alteradas para permitir a retenção da forma durante a desmoldagem. O que é necessário nesta etapa é transformar a lama viscosa em um corpo elástico sem remoção da fase líquida. A ideia básica é evitar, a todo custo, passar por uma fase onde exista uma interface líquido/vapor. A presença de interfaces líquido/vapor pode resultar em forças capilares fortes que podem causar rearranjo de partículas e aglomeração. E embora isso seja desejável durante a sinterização em fase líquida, é indesejável quando uma lama é seca porque é incontrolável e pode resultar em tensões de encolhimento, que por sua vez podem resultar na formação de aglomerados ou grandes rachaduras entre áreas que encolhem em taxas diferentes.

Outra possível fonte de falhas pode ser introduzida durante a prensagem a frio de pós aglomerados como resultado de diferenças de densidade entre os aglomerados e a matriz. Quando a pressão é removida, a dilatação elástica dos aglomerados e da matriz pode ser suficientemente diferente para causar a formação de rachaduras; isto é, a recuperação elástica dos aglomerados será diferente da matriz como resultado das diferenças em sua densidade.

\section*{Resumo}

\begin{itemize}
    \item Variações locais na curvatura resultam na transferência de massa de áreas de curvatura positiva (convexas) para áreas de curvatura negativa (côncavas). Quantitativamente, esse diferencial de potencial químico é dado por esta equação, que deve ser positiva se a sinterização for ocorrer.
    \item Na escala atômica, esse gradiente de potencial químico resulta em um aumento local na pressão parcial do sólido e uma diminuição local na concentração de vacância nas áreas convexas em relação às áreas côncavas. Visto de outra perspectiva, a matéria será sempre deslocada dos picos para os vales.
    \item Pressões de vapor altas e partículas pequenas tenderão a favorecer mecanismos de transporte de gases que levam ao crescimento de grão, enquanto pressão de vapor baixa e difusividades rápidas de massa ou fronteiras de grãos tenderão a favorecer a densificação. Se o fluxo atômico for da superfície das partículas para a região do pescoço, ou da superfície de partículas menores para maiores, isso leva, respectivamente, ao crescimento do pescoço e à coalescência. No entanto, se os átomos difundirem da área da fronteira do grão para a região do pescoço, resultará em densificação. Portanto, todos os modelos que invocam a contração assumem invariavelmente que as áreas da fronteira do grão ou as superfícies livres são os sumidouros de vacância e que as superfícies do pescoço são as fontes de vacância.
    \item A cinética de sinterização depende do tamanho das partículas e dos valores relativos dos coeficientes de transporte, com partículas menores favorecendo a difusão da fronteira do grão e da superfície e partículas maiores favorecendo a difusão em massa.
    \item Durante o estágio intermediário de sinterização, a porosidade é eliminada pela difusão de vacâncias de áreas porosas para fronteiras de grãos, superfícies livres ou deslocamentos. A uniformidade do empacotamento de partículas e a ausência de aglomerados são importantes para a realização de uma densificação rápida.
    \item Nas últimas etapas da sinterização, o objetivo geralmente é eliminar os últimos resquícios de porosidade. Isso só pode ser realizado, no entanto, se os poros permanecerem presos às fronteiras do grão. Uma maneira de fazer isso é diminuir a mobilidade da fronteira do grão por dopagem ou pela adição de inclusões ou segundas fases nas fronteiras.
    \item Durante a sinterização em fase líquida, as forças capilares que se desenvolvem podem ser bastante grandes. Isso resulta no rearranjo das partículas, bem como aumenta a dissolução de matéria entre elas, resultando em encolhimento rápido e densificação. A maioria das cerâmicas comerciais é fabricada por algum tipo de sinterização em fase líquida. Existem três estágios na sinterização em fase líquida: o rearranjo de partículas, a reprecipitação da solução (por capilaridade), e a sinterização em estado sólido.
    \item A aplicação de uma força externa a um compacto de pó durante a sinterização pode aumentar consideravelmente a cinética de densificação, aumentando os gradientes de potencial químico dos átomos entre as partículas, induzindo-os a migrar para longe dessas áreas.
\end{itemize}
