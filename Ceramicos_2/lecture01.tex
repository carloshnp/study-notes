\lecture{0.1}{}{Propriedades Físicas de Cerâmicos}

\section*{Propriedades Físicas}

\begin{definition}[Densidade]
  Densidade ($\rho$) é uma medida da massa ($m$) por unidade de volume ($V$) de um material. É dada em $g/cm^3$. A densidade é influenciada por:
  \begin{itemize}
    \item Tamanho e peso atômico dos elementos
    \item Empacotamento atômico na estrutura cristalina
    \item Quantidade de porosidade na microestrutura
  \end{itemize}
\end{definition}

Os tipos de densidade são dados por:

\begin{definition}[Densidade cristalográfica]
  A densidade cristalográfica é a densidade ideal de uma estrutura cristalina específica é calculada pelos dados da composição química e pelo espaçamento interatômico obtido pela difração de raios-x. É calculada dividindo a massa de uma célula unitária do material pelo volume da célula unitária.
\end{definition}
\begin{eg}
  Densidade de um elemento com estrutura CFC:
  \begin{gather}
    m = \frac{\text{nº de átomos por célula} \times m_{\text{átomo}}}{\text{nº de avogadro}} \\
    \rho = \frac{m}{V_{\text{célula unitária}}}
  \end{gather}
\end{eg}

As massas atômicas dos elementos que estão na célula unitária tem um efeito majoritário na densidade cristalográfica do material, assim como o empacotamento. Com maior empacotamento, há uma maior densidade cristalográfica. Outros fatores como polimorfismo podem afetar a densidade de materiais com os mesmos elementos.

\begin{definition}[Densidade teórica]
  A densidade teórica é a densidade de um material que contém zero porosidade microestrutural, levando em consideração múltiplas fases, defeitos estruturais e solução sólida. A densidade teórica pode ser calculada a partir da densidade cristalográfica de da fração volumétrica de cada fase sólida na microestrutura.
  \begin{gather}
    \rho_{\text{teórica}} = \sum V_{f_n} \rho_n
  \end{gather}
  Por exemplo, podemos calcular a porosidade a partir das densidades teórica e bulk:
  \begin{gather}
    \% \rho_{\text{teórica}} = \frac{\rho_{\text{bulk}}}{\rho_{\text{teórica}}} \times 100 \\
    \% \text{porosidade} = 100\% - \% \rho_{\text{teórica}}
  \end{gather}
\end{definition}

\begin{definition}[Densidade bulk]
  Densidade em massa ou volumétrica (bulk density) é a densidade medida de um corpo cerâmico bruto (bulk ceramic body), levando em consideração toda a porosidade, defeitos de rede e fases. Os materiais cerâmicos comerciais costumam possuir mais de uma fase cristalina, e geralmente uma fase não-cristalina. Cada uma dessas fases possue uma densidade diferente dependendo do empacotamento e dos átomos presentes, além da porosidade. Portanto, definimos a densidade \textit{bulk} como:
  \begin{gather}
    \begin{align}
      \rho_{\text{bulk}} & = \frac{m}{V_{\text{bulk}}} \\
                      & = \frac{m}{V_{\text{sólido}} + V_{\text{poros}}}
    \end{align}
  \end{gather}
\end{definition}

A densidade \textit{bulk} de um material cerâmico com geometria complexa pode ser medido pelo princípio de Arquimedes, ou de líquidos pesados calibrados para cerâmicos pequenos que não possuem porosidade aberta.

\begin{definition}[Gravidade específica]
  A gravidade específica é a densidade de um material relativa à densidade de um volume igual de água à $4^{\circ}C$ (adimensional).
  \begin{gather}
    SG = \frac{\rho_{\text{material}}}{\rho_{\text{água à 4ºC}}}
  \end{gather}
\end{definition}

A porosidade aberta de um material refere-se à fração volumétrica total de um cerâmico que consiste em poros interconectaros e abertos, que são acessíveis a fluidos ou gases, e não estão isolados do meio, expressa como uma porcentagem do volume de poros em relação ao volume total do material. Essa medida afeta diretamente propriedades como a permeabilidade de gases e líquidos, a resistência mecânica e suas propriedades térmicas.

\paragraph*{Comportamento no derretimento} Cerâmicos podem ter características diferentes de fusão, de forma congruente ou incongruente, ou alguns podem sublimar enquanto outros se decompõem. Essas tendências são altamente determinadas pelas forças das ligações atômicas. Os materiais podem seguir as seguintes tendências:

\begin{itemize}
  \item Materiais com ligações atômicas e estruturais primárias fortes tendem a possuir altos pontos de fusão.
  \item Metais de transição mais fortemente ligados ($Fe$, $Ni$, $Co$, etc.) possuem pontos de fusão muito mais altos.
  \item Materiais cerâmicos iônicos multivalentes com maior comportamento covalente possuem maiores pontos de fusão.
  \item Materiais cerâmicos covalentes fortemente ligados possuem ponto de fusão ou temperaturas de dissociação muito altos.
  \item Materiais fortemente ligados como $W$, $Ta$ e $Mo$ possuem altíssimos pontos de fusão.
  \item Materiais com ligações primárias fracas ou com ligações de van der Waals como ligação estrutural principal possuem baixos pontos de fusão.
  \item Materiais com metais alcalinos fracamente ligados e cerâmicos iônicos monovalentes possuem baixos pontos de fusão.
  \item Materiais orgânicos possuem baixos pontos de fusão ou temperaturas de decompoição por conta das fracas ligações de van der Waals entre as moléculas.
  \item Estruturas lineares como termoplásticos derretem, enquanto cadeias ramificadas como resinas tendem a decompor ou a degradar.
  \item Ligações cruzadas ou ramificações tendem a aumentar a temperatura de fusão de composições termoplásticas.
\end{itemize}

\paragraph*{Medida de temperatura de fusão} Para a medida da temperatura de fusão ou decomposição, temos:

\begin{itemize}
  \item Em baixas temperaturas, basta utilizar um forno com mecanismo de aquecimento e um termômetro calibrado para a medida.
  \item Acima de temperaturas de 1700ºC é necessário utilizar pirômetros óticos.
  \item Para temepraturas acima de 2000ºC, é necessário utilizar lasers de alta potência para fundir os cerâmicos, e utilizar lasers de baixa potência para detectar, por reflexão, ondulações de ondas líquidas formadas na superfície durante a fusão.
\end{itemize}

\section*{Propriedades térmicas}

\begin{definition}[Capacidade térmica]
  A capacidade térmica, ou capacidade de calor $c$ de um material é a energia necessária para aumentar sua temperatura, especificamente, a quantidade de calor necessária para aumentar a temperatura de uma substância por um grau.
\end{definition}

De forma análoga, a capacidade térmica molar de um material é a quantidade de calor necessária para aumentar a temperatura de uma massa molecular de um material por um grau. Para a maioria dos materiais cerâmicos, a capacidade térmica aumenta de zero à 0K, até um valor de $6,0 \text{cal/g atom.ºC}$, em torno de 1000ºC.

A capacidade térmica de um material é relativamente insensívele à estrutura cristalina ou à composição do material. Porém, um dos fatores principais que afeta a capacidade térmica é a porosidade de um material. Um material cerâmico altamente poroso possui menos material sólido por unidade de volume do que um material sem poros, e um material sem poros necessita de mais energia de calor para aquecer à uma temperatura específica do que um material com muitos poros.

Já o calor específico de um material é a taxa de capacidade térmica de um material em relação à água à 15ºC, sendo um valor adimensional.

A capacidade térmica de um mateiral é determinada pelos efeitos da temperatura na:

\begin{itemize}
  \item A energia vibracional e rotacional dos átomos nos materiais.
  \item A variação no nível de energia dos elétrons na estrutura.
  \item A mudança nas posições atômicas durante a formação de defeitos de rede (vacâncias ou interstícios), transições ordenadas e desordenadas, orientação magnética e transformações polimórficas.
\end{itemize}


