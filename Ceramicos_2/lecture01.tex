\lecture{0.1}{}{Propriedades Físicas e Fabricação de Cerâmicos}

\section*{Propriedades Físicas}

\begin{definition}[Densidade]
  Densidade ($\rho$) é uma medida da massa ($m$) por unidade de volume ($V$) de um material. É dada em $g/cm^3$. A densidade é influenciada por:
  \begin{itemize}
    \item Tamanho e peso atômico dos elementos
    \item Empacotamento atômico na estrutura cristalina
    \item Quantidade de porosidade na microestrutura
  \end{itemize}
\end{definition}

Os tipos de densidade são dados por:

\begin{definition}[Densidade cristalográfica]
  A densidade cristalográfica é a densidade ideal de uma estrutura cristalina específica é calculada pelos dados da composição química e pelo espaçamento interatômico obtido pela difração de raios-x. É calculada dividindo a massa de uma célula unitária do material pelo volume da célula unitária.
\end{definition}
\begin{eg}
  Densidade de um elemento com estrutura CFC:
  \begin{gather}
    m = \frac{\text{nº de átomos por célula} \times m_{\text{átomo}}}{\text{nº de avogadro}} \\
    \rho = \frac{m}{V_{\text{célula unitária}}}
  \end{gather}
\end{eg}

As massas atômicas dos elementos que estão na célula unitária tem um efeito majoritário na densidade cristalográfica do material, assim como o empacotamento. Com maior empacotamento, há uma maior densidade cristalográfica. Outros fatores como polimorfismo podem afetar a densidade de materiais com os mesmos elementos.

\begin{definition}[Densidade teórica]
  A densidade teórica é a densidade de um material que contém zero porosidade microestrutural, levando em consideração múltiplas fases, defeitos estruturais e solução sólida. A densidade teórica pode ser calculada a partir da densidade cristalográfica de da fração volumétrica de cada fase sólida na microestrutura.
  \begin{gather}
    \rho_{\text{teórica}} = \sum V_{f_n} \rho_n
  \end{gather}
  Por exemplo, podemos calcular a porosidade a partir das densidades teórica e bulk:
  \begin{gather}
    \% \rho_{\text{teórica}} = \frac{\rho_{\text{bulk}}}{\rho_{\text{teórica}}} \times 100 \\
    \% \text{porosidade} = 100\% - \% \rho_{\text{teórica}}
  \end{gather}
\end{definition}

\begin{definition}[Densidade bulk]
  Densidade em massa ou volumétrica (bulk density) é a densidade medida de um corpo cerâmico bruto (bulk ceramic body), levando em consideração toda a porosidade, defeitos de rede e fases. Os materiais cerâmicos comerciais costumam possuir mais de uma fase cristalina, e geralmente uma fase não-cristalina. Cada uma dessas fases possue uma densidade diferente dependendo do empacotamento e dos átomos presentes, além da porosidade. Portanto, definimos a densidade \textit{bulk} como:
  \begin{gather}
    \begin{align}
      \rho_{\text{bulk}} &= \frac{m}{V_{\text{bulk}}} \\
              &= \frac{m}{V_{\text{sólido}} + V_{\text{poros}}} 
    \end{align}
  \end{gather}
\end{definition}

A densidade \textit{bulk} de um material cerâmico com geometria complexa pode ser medido pelo princípio de Arquimedes, ou de líquidos pesados calibrados para cerâmicos pequenos que não possuem porosidade aberta.

\begin{definition}[Gravidade específica]
  A gravidade específica é a densidade de um material relativa à densidade de um volume igual de água à $4^\circ C$ (adimensional).
  \begin{gather}
    SG = \frac{\rho_{\text{material}}}{\rho_{\text{água à 4ºC}}}
  \end{gather}
\end{definition}

A porosidade aberta de um material refere-se à fração volumétrica total de um cerâmico que consiste em poros interconectaros e abertos, que são acessíveis a fluidos ou gases, e não estão isolados do meio, expressa como uma porcentagem do volume de poros em relação ao volume total do material. Essa medida afeta diretamente propriedades como a permeabilidade de gases e líquidos, a resistência mecânica e suas propriedades térmicas.

