\lecture{0.2}{}{Fabricação de Materiais Cerâmicos}

\section*{Métodos gerais para processamento de cerâmicos}
Os cerâmicos podem ser processados por uma ampla variedade de abordagens, dependendo da microestrutura e propriedades desejadas. Cerâmicas convencionais começam com um pó fino que é compactado e densificado em alta temperatura para alcançar uma microestrutura uniforme de grão fino, de preferência com um mínimo de porosidade e inclusões de segunda fase. Em contraste, a maioria dos materiais refratários requer uma mistura de fases e tamanhos de partículas ou agregados e frequentemente com porosidade controlada. O vidro é processado em estado fundido com o objetivo de obter um produto transparente que seja completamente livre de poros e inclusões. Compósitos de matriz cerâmica (CMCs) envolvem uma segunda fase (partículas, fios ou fibras) que muitas vezes deve estar alinhada e não deve ser degradada durante a parte de densificação do processo.

\paragraph{Compactação de pós}

Os pós iniciais geralmente não possuem tamanho ou distribuição de tamanho adequados para os passos de processamento subsequentes. Eles exigem etapas como moagem ou calcinação, seguidas de medições de controle de qualidade para verificar se o dimensionamento desejado foi alcançado.

O próximo passo, de pré-consolidação, resulta em preparar os pós para o posterior passo de consolidação de conformação. Essa preparação difere para cada tipo de método de conformação. Para a compactação por prensagem, alguma porcentagem de água ou um ligante polimérico solúvel são adequados. Para moldagem por deslizamento e moldagem em fita, cerca de 50\% em volume de líquido, além de agentes umectantes, dispersantes ou ligantes, são necessários para fornecer uma mistura fluída. Para extrusão e moldagem por injeção, diferentes aditivos são necessários para produzir uma mistura viscosa que requer pressão para forçar a mistura para dentro (ou através) de ferramentas de metal.

As misturas de pó/liquido/ligante são cuidadosamente medidas em relação a fatores como a taxa de compactação e a viscosidade, dependendo do método de conformação selecionado. O objetivo na etapa de conformação é compactar o pó para que as partículas sejam uniformemente distribuídas e geralmente o mais próximas possível do empacotamento mais denso, resultando preferencialmente em um corpo verde (não queimado) com porosidade de no máximo 50\%.

Uma vez que a compactação é concluída, os líquidos e ligantes são removidos em baixas temperaturas e o corpo verde está pronto para o passo de densificação. A densificação é alcançada em uma etapa de alta temperatura conhecida como sinterização. Durante a sinterização, o material difunde entre as partículas nos pontos de contato ou é aprimorado pela presença de uma fase líquida. Partículas pequenas são consumidas por partículas maiores, os poros são reduzidos ou eliminados e uma microestrutura policristalina se forma. A porosidade do corpo verde é lentamente reduzida para geralmente menos de 2\%, acompanhada por uma quantidade equivalente de contração (~50\% em volume, ~17\% linear) da peça cerâmica. Muitas peças cerâmicas requerem tolerâncias próximas, portanto, a contração e distorção durante a sinterização devem ser controladas.

Para a maioria das aplicações, deseja-se uma microestrutura uniforme de grão fino. A microestrutura resultante depende muito da natureza dos pós iniciais, do tamanho e distribuição de tamanho das partículas, do empacotamento de partículas alcançado durante a consolidação e do tempo e temperatura de sinterização.

% \paragraph*{Processamento refratário}