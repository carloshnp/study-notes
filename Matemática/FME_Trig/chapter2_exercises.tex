\lecture{2x}{}{Exercises (FME 3 - 9th ed.)}

 \paragraph{2.1}

 \begin{gather}
  \begin{align}
    sin(B) = \frac{9}{15} \\
    cos(B) = \frac{12}{15} \\
    tan(B) = \frac{9}{12} \\
    cot(B) = \frac{12}{9} \\
    sin(C) = \frac{12}{15} \\
    cos(C) = \frac{9}{15}\\
    tan(C) = \frac{12}{9} \\
    cot(C) = \frac{9}{12}
  \end{align}
 \end{gather}

 \paragraph{2.2}

 \begin{gather}
    a^2 = b^2 + c^2 \rightarrow a = \sqrt{4^2 + 2^2} = \sqrt{20} = 2 \sqrt{5} \\
    sin(D) = \frac{4}{2 \sqrt{5}} = \frac{2}{\sqrt{5}} = \frac{2 \sqrt{5}}{5} \\
    cos(D) = \frac{2}{2 \sqrt{5}} = \frac{\sqrt{5}}{5} \\
    tan(D) = \frac{4}{2} = 2 \\
    cot(D) = \frac{2}{4} = \frac{1}{2} \\ 
    sin(E) = \frac{2}{2 \sqrt{5}} = \frac{\sqrt{5}}{5} \\
    cos(E) = \frac{4}{2 \sqrt{5}} = \frac{2 \sqrt{5}}{5} \\
    tan(E) = \frac{2}{4} = \frac{1}{2}
    cot(E) = \frac{4}{2} = 2
 \end{gather}

 \paragraph{2.3}

 \begin{gather}
   a^2 = b^2 + c^2 \rightarrow c^2 = a^2 - c^2 \rightarrow c = \sqrt{(2 \sqrt{3})^2 - 3^2} = \sqrt{3} \\
   sin(B) = \frac{3}{2 \sqrt{3}} = \frac{\sqrt{3}}{2} \\
   cos(B) = \frac{\sqrt{3}}{2 \sqrt{3}} = \frac{1}{2} \\
   tan(B) = \frac{3}{\sqrt{3}} = \sqrt{3} \\
   cot(B) = \frac{\sqrt{3}}{3} \\
   sin(C) = \frac{\sqrt{3}}{2 \sqrt{3}} = \frac{1}{2} \\
   cos(C) = \frac{3}{2 \sqrt{3}} = \frac{\sqrt{3}}{2} \\
   tan(C) = \frac{\sqrt{3}}{3} \\
   cot(C) = \sqrt{3}
 \end{gather}

 \paragraph{2.4}

 \begin{gather}
   \bar{AB} = \frac{4}{5} \times 50 = 40 \\
   a^2 = b^2 + c^2 \rightarrow c = \sqrt{50^2 - 40^2} = \sqrt{900} = 30 = \bar{AC} 
 \end{gather}

 \paragraph{2.5}

 \begin{gather}
  cos(B) = \frac{\bar{AB}}{\bar{BC}} \\
  \begin{align}
    \bar{AB} &= \frac{2 \sqrt{51}}{17} \times 2 \sqrt{17} \\
             &= \frac{2 \sqrt{17 \times 3}}{17} \times 2 \sqrt{17} \\
             &= \frac{2 \sqrt{3}}{\sqrt{17}} \times 2 \sqrt{17} \\
             &= 4 \sqrt{3}       
  \end{align} \\
  c^2 = (\bar{AB})^2 - (\bar{BC})^2 \rightarrow c = \sqrt{(2 \sqrt{17})^2 - (4 \sqrt{3})^2} = 2 \sqrt{5} 
 \end{gather}

 \paragraph{2.6}

 \paragraph{2.7}

 \begin{gather}
  sin(B) = \frac{3}{5} \\
  \begin{align}  
    cos(B) \rightarrow cos^2(B) = 1 - sin^2(B) \\
    cos(B) = \sqrt{1^2 - (\frac{3}{5})^2} \\
    cos(B) = \sqrt{1 - \frac{9}{25}} \\
    cos(B) = \sqrt{\frac{16}{25}} \\
    cos(B) = \frac{4}{5}
  \end{align}
  \begin{align}
    tan(B) = \frac{sin(B)}{cos(B)} \\
    tan(B) = \frac{\frac{3}{5}}{\frac{4}{5}} \\
    tan(B) = \frac{3}{4}
  \end{align}
  \begin{align}
    cot(B) = \frac{1}{tan(B)} \\
    cot(B) = \frac{1}{\frac{3}{4}} \\
    cot(B) = \frac{4}{3}
  \end{align}
 \end{gather}

 \begin{gather}
   sin(B) = \frac{2}{3} \\
   \begin{align}
     cos(B) = \sqrt{1^2 - (\frac{2}{3})^2} \\
     cos(B) = \sqrt{1 - (\frac{4}{9})}
     cos(B) = \frac{\sqrt{5}}{3}
   \end{align}
   \begin{align}
     tan(B) = \frac{\frac{2}{3}}{\frac{\sqrt{5}}{3}} \\
     tan(B) = \frac{2}{\sqrt{5}} \left( \frac{\sqrt{5}}{\sqrt{5}} \right) \\
     tan(B) = \frac{2 \sqrt{5}}{5} 
   \end{align}
   \begin{align}
    cot(B) = \frac{1}{\frac{2 \sqrt{5}}{5}} \\
    cot(B) = \frac{5}{2 \sqrt{5}} \left( \frac{\sqrt{5}}{\sqrt{5}} \right) \\
    cot(B) = \frac{\sqrt{5}}{2} 
   \end{align}
 \end{gather}

 \begin{gather}
   sin(B) = 0,57 \\
  \begin{align}
    cos(B) &= \sqrt{1^2 - (0,57)^2} \\
           &= \sqrt{0,6751} \\
           &= 0,82
  \end{align}
  \begin{align}
    tan(B) &= \frac{0,57}{0,82} \\
           &= 0,69
  \end{align}
  \begin{align}
    cot(B) &= \frac{1}{0,69} \\
           &= 1,45
  \end{align}
 \end{gather}

 \begin{gather}
  sin(B) = 0,95 \\
  \begin{align}
    cos(B) &= \sqrt{1^2 - (0,95)^2} \\
           &= \sqrt{1 - 0,9} \\
           &= 0,32
  \end{align}
  \begin{align}
    tan(B) &= \frac{0,95}{0,32} \\
           &= 2,97 
  \end{align}
  \begin{align}
    cot(B) &= \frac{1}{2,97} \\
           &= 0,33
  \end{align}
 \end{gather}

 \paragraph{2.8}



