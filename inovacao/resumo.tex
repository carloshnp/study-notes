\chapter*{O Projeto Genoma Humano como bolha disruptiva}

\paragraph{} O Projeto Genoma Humano, concebido na década de 80, iniciado formalmente em 1990, e concluído em 2003, apresenta-se como uma bolha social, onde definido por Monika Gisler, de forma análoga à bolha financeira, provou-se uma transição exorbitante na biotecnologia de forma a revolucionar a biologia e medicina no século atual, através da motivação do suporte entusiástico entre o setor público e privado, com riscos extraordinários e impossíveis de serem racionalizados numa análise de custo-benefício, a fim de gerar frutos que foram e ainda serão aproveitados por décadas.

\paragraph{} Compreende-se que o projeto recebeu, simultaneamente, grande apoio por diferentes atores (incluíndo o público), criação de crédito pelo investimento público e privado, proliferação de empreendimentos, aceleração rápida do preço das firmas correspondentes através do mercado de ações, e a saturação e terminação abrupta do programa. Desta forma, o entusiasmo pela determinação completa do mapeamento das sequência de pares de base nitrogenadas que formam o DNA humano, permitiu a proliferação de mais de 300 firmas interessadas no projeto para desenvolver a tecnologia genômica, catalizando o desenvolvimento de fármacos, a agricultura e outros setores industriais para métodos baseados no DNA nos dias atuais.

\paragraph{} Desta forma, o presente ensaio tem como objetivo compreender a inovação disruptiva gerada pelo projeto, as transformações causadas, as suas fontes, e a proposta de valor que levou ao desenvolvimento da tecnologia em tempo recorde, bem como os aspectos econômicos e sociais que foram alterados durante e após o projeto.


