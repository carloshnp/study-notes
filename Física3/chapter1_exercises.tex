\lecture{1x}{}{Exercises (Halliday 11th ed.)}

\paragraph{Question 1}

Temos que a força eletrostática é dada por:

\[
  F = \frac{1}{4 \pi \epsilon_0} \frac{|q_1| |q_2|}{r^2}
\]

Considerando que a carga do meio é positiva, temos que:

\begin{itemize}
  \item Para a carga 1, as cargas da extremidade se cancelam, enquanto as cargas mais próximas geram uma força líquida para a direita.
  \item Para a carga 2, as cargas mais próximas se cancelam, enquanto as cargas da extremidade geram uma força líquida para a esquerda.
  \item Para a carga 3, todas as cargas se somam com uma força líquida para a direita.
  \item Para a carga 4, todas as cargas se cancelam, então a força líquida é igual a 0.
\end{itemize}

Considerando que a força eletrostática é inversamente proporcional à distância, temos que:
\begin{gather*}
  F_3 > F_1 > F_2 > F_4
\end{gather*}

\paragraph{Question 2}
Segundo a lei de conservação das cargas, quando as esferas se tocarem e forem separadas, elas devem possuir a mesma carga, o que resulta na metade da carga líquida em cada esfera. Portanto temos:

\subparagraph{Par 1}
\begin{gather}
  \begin{align}
    Q_{sum} = Q_1 + Q_2 = +6e -4e = +2e \\
    Q_{mean} = \frac{Q_{sum}}{2} = \frac{+2e}{2} = +e
  \end{align}
\end{gather}

\subparagraph{Par 2}
\begin{gather}
  \begin{align}
    Q_{sum} = 0 +2e = +2e \\
    Q_{mean} = \frac{1}{2} +2e = +e 
  \end{align}
\end{gather}

\subparagraph{Par 3}
\begin{gather}
  \begin{align}
    Q_{\text{sum}} = -12e +14e = +2e \\
    Q_{\text{mean}} = +e
  \end{align}
\end{gather}

A magnitude de transferência foi:
\[
  3 > 1 > 2
\]

Já a carga restante na esfera carregada positivamente foi igual em todos os pares.
