\lecture{1}{}{Fenômenos de superfície}

\begin{multicols*}{2}

  \section*{Revisão de termodinâmica}

  \paragraph{ }

  \begin{definition}[1ª Lei - Conservação de energia]
    Em um processo termodinâmico fechado, a alteração da energia interna depende do calor acumulado pelo sistema e a alteração do trabalho realizado.
    \begin{gather}
      dU = dq + dW
    \end{gather}
  \end{definition}

  \begin{corollary}
    Para trabalhos reversíveis de volume, temos que:
    \begin{gather}
      dW = -p dV \\
      dU = dq - p dV
    \end{gather}
    Para processos reversíveis, temos que:
    \begin{gather}
      dq = TdS \\
      dU = TdS - pdV \ \text{(equação fundamental)}
    \end{gather}
  \end{corollary}
  
  Portanto a energia interna de um sistema fechado é função da entropia (distribuição de energia no sistema) e do volume (distância média entre as partículas), ou seja:
  \begin{theorem}[Energia interna como função da entropia e volume]
    \begin{gather}
      U = U(S,V) \\
      dU = \left( \frac{\partial U}{\partial S} \right)_V dS + \left( \frac{\partial U}{\partial V}  \right)_S dV \\
      \left( \frac{\partial U}{\partial S}  \right)_V = T \ , \ \left( \frac{\partial U}{\partial V}  \right)_S = -p
    \end{gather}  
  \end{theorem}

  Em sistemas abertos, há a troca de massa com as vizinhanças, portanto a energia pode ser alterada a partir da variação das composições das espécies, onde $n_i$ é o número de moles do sistema.
  \begin{theorem}[Energia interna em sistemas abertos como função]
    \begin{gather}
      U = U(S,V,n_1,n_2, \cdots)    
    \end{gather}
  \end{theorem}

  A entalpia também pode ser definida como função:
  \begin{definition}[Entalpia como função]
    Pela definição de entalpia, temos:
    \begin{gather}
      H = U + pV \\
      dH = dU + pdV + Vdp
    \end{gather}
    Substituindo a energia interna, temos:
    \begin{gather}
      dH = TdS - pdV + pdV + Vdp \\
      dH = TdS + Vdp \ \text{(equação fundamental)}
    \end{gather}
    Portanto, temos a entalpia em função de:
    \begin{gather}
      H = H(S,p) \ \text{(sistemas fechados)} \\
      H = H(S,p,n_1,n_2, \cdots) \ \text{(sistemas abertos)}
    \end{gather}
  \end{definition}

  A energia livre de Gibbs também pode ser definida da mesma forma:

  \begin{definition}[Energia livre de Gibbs como função]
    Pela definição, temos:
    \begin{gather}
      G = H - TS \\
      dG = dH - TdS - SdT \\
    \end{gather}
    Substituindo a entalpia, temos:
    \begin{gather}
      dG = TdS + Vdp - TdS - SdT \\
      dG = -SdT + Vdp \ \text{(equação fundamental)}
    \end{gather}
    Portanto, temos a energia de Gibbs em função de:
    \begin{gather}
      G = G(T,p) \ \text{(sistemas fechados)} \\
      G = G(T,p,n_1,n_2,\cdots) \ \text{(sistemas abertos)}
    \end{gather}
  \end{definition}

  Com a variação dos moles das espécies presentes, temos que a diferencial total da energia livre de Gibbs é:
  \begin{definition}[Diferencial da energia de Gibbs com variação dos moles]
    \begin{gather}
      \begin{align}
        dG = & \left( \frac{\partial G}{\partial T}  \right)_{p, n_1, n_2, \cdots} dT + \\
             & \left( \frac{\partial G}{\partial p}  \right)_{T, n_1, n_2, \cdots} dp + \\
             & \left( \frac{\partial G}{\partial n_1}  \right)_{T, p, n_2, \cdots} dn_1 + \\
             & \left( \frac{\partial G}{\partial n_2}  \right)_{T, p, n_1, \cdots} dn_2 + \cdots
      \end{align}
    \end{gather}    
  \end{definition}

  \begin{corollary}
    Se $dn_i = 0$, a expressão se torna:
    \begin{gather}
      dG = \left( \frac{\partial G}{\partial T}  \right)_{p, n_1, n_2, \cdots} dT + \left( \frac{\partial G}{\partial p}  \right)_{T, n_1, n_2, \cdots} dp
    \end{gather}
    Assim, temos que:
    \begin{gather}
      \left( \frac{\partial G}{\partial T}  \right)_{p, n_1, n_2, \cdots} = -S \\
      \left( \frac{\partial G}{\partial p}  \right)_{T, n_1, n_2, \cdots} = V
    \end{gather}
    Portanto, podemos reescrever a equação de Gibbs na forma diferencial como: 
    \begin{gather}
      \begin{align}
        dG = & -SdT + Vdp + \\
             & \left( \frac{\partial G}{\partial n_1}  \right)_{T, p, n_2, \cdots} dn_1 + \\
             & \left( \frac{\partial G}{\partial n_2}  \right)_{T, p, n_1, \cdots} dn_2 + \cdots
      \end{align}
    \end{gather}
  \end{corollary}

  \begin{theorem}[Potencial químico]
    Se definirmos a variação da energia livre de Gibbs com o número de moles como sendo o potencial químico, temos que:
    \begin{gather}
      \left( \frac{\partial G}{\partial n_1}  \right)_{T, p, n_2, \cdots} = \mu_1 \\
      \left( \frac{\partial G}{\partial n_2}  \right)_{T, p, n_1, \cdots} = \mu_2
    \end{gather}
    E portanto, para um sistema de composição variável:
    \begin{gather}
      \begin{align}
        dG &= -SdT + Vdp + \mu_1 dn_1 + \mu_2 dn_2 + \cdots \\
           &= -SdT + Vdp + \sum \mu_i dn_i
      \end{align}
    \end{gather}
  \end{theorem}
  \begin{remark}
    Se $p$ e $T$ são mantidas constante, a variação de $G$ indica se o sistema está em equilíbrio, ou sofrendo transformação espontânea, ou sofrendo transformação não-espontânea.
  \end{remark}

  Considerando $p$ e $T$ constantes, e retirando $dn$ moles de $a$ e introduzindo $dn$ moles no ponto $b$, temos duas variações na energia de Gibbs:
  \begin{theorem}[Potencial químico e transporte de massa]
    \begin{gather}
      dG^a = - \mu^a dn \ (\Delta G < 0) \\
      dG^b = + \mu^b dn \ (\Delta G > 0)
    \end{gather}
    Portanto, o potencial químico está associado ao transporte de massa da região de maior potencial químico para a região de menor potencial químico.
    Para que um sistema de composição variável esteja em equilíbrio, é necessário que, além da pressão e temperatura constantes, os potenciais químicos dos componentes do sistema também fiquem constantes.
  \end{theorem}

  \begin{definition}[Equação de Gibbs-Duhem]
    Considerando um sistema binário com temperatura e pressão constantes, temos:
    \begin{gather}
      dG = -SdT + Vdp + \mu_1 dn_1 + \mu_2 dn_2 \\
      dG = \mu_1 dn_1 + \mu_2 dn_2 \\
      G = \mu_1 n_1 + \mu_2 n_2
    \end{gather}
    Considerando a diferencial da energia de Gibbs, temos que:
    \begin{gather}
      dG = \mu_1 dn_1 + n_1 d\mu_1 + \mu_2 dn_2 + n_2 d\mu_2 \\
      n_1 d\mu_1 + n_2 d\mu_2 = 0 \\
      \sum n_i d\mu_i = 0 \ \text{(equação de Gibbs-Duhem)}
    \end{gather}
    Portanto, a equação de Gibbs-Duhem nos diz que os potenciais químicos de componentes de uma mistura não podem variar de forma independente.
  \end{definition}

  \section*{Fenômenos de Interface}

  \paragraph{ } Para definirmos os fenômenos interfaciais, precisamos entender a variação de energia que ocorre na superfície. Portanto, é fundamental definirmos a energia superficial de um sistema. Desta forma, podemos relacionar a energia superficial com o excesso de concentração dos solutos e com o potencial químico, o que nos permite explicar o conceito de adsorção.

  \begin{definition}[Energia superficial]
    Para um sistema aberto de composição variável, $G = G(T,p,n_i)$. Já para a superfície de uma solução, $G^{A}=G(T,p,n_i,A)$, onde $A$ é a área da superfície. Portanto, para aumentar a superfície, temos que levar moléculas do seio da solução para a superfície, e como esse não é um processo espontâneo, precisamos fornecer energia para o sistema. Dessa forma, temos que:
    \begin{gather}
      \begin{align}
        dG^{A} = & \left( \frac{\partial G}{\partial n_1}  \right)_{T,p,n_2, \cdots,A} dn_1 + \\
               & \left( \frac{\partial G}{\partial n_2}  \right)_{T,p,n_1,\cdots,A} dn_2 + \\
               & \cdots + \\
               & \left( \frac{\partial G}{\partial A} \right)_{T,p,n_1} dA
      \end{align}
    \end{gather}

    \begin{gather}
      \begin{aligned}
        dG^{A} = \mu_1^{A} dn_1^{A} + \mu_2^{A} dn_2^{A} + \cdots + \gamma dA 
      \end{aligned}
    \end{gather}
    Desta forma, temos $\gamma$, que é a energia superficial, ou seja, a energia por unidade de área da superfície. Portanto:
    \begin{gather}
      \gamma = \frac{E}{A} = \frac{F d}{d^2} = \frac{F}{d}
    \end{gather}
    Logo, $\gamma$ é dada por $erg/cm^2$ ou $dinas/cm$. 
  \end{definition}
  \begin{remark}
    Quando exprimimos $\gamma$ como energia por área, chamamos de \textbf{energia superficial}, e quando exprimimos $\gamma$ como força por comprimento chamamos de \textbf{tensão superficial}.
  \end{remark}

\end{multicols*}
