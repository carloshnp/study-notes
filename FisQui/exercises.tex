% \noalign{\smallskip} \hline \noalign{\smallskip}

\lecture{X}{}{Exercícios}

\section*{Lista 2 - Equilíbrio químico}

\paragraph*{2.1}

\begin{gather}
    \begin{align}
        AgCl & \leftrightharpoons Ag^+ + Cl^- \text{(em equilíbrio com corpo de fundo)} \\
        NaCl & \leftrightharpoons Na^+ + Cl^-
    \end{align}
\end{gather}

Como a solução já está saturada, o $NaCl$ não irá alterar a concentração de cloro, pois já há saturação do íon $Cl^-$. Portanto, haverá um equilíbrio na dissociação das espécies (para aportar a dissociação de $NaCl$), e um consequente aumento do corpo de fundo de $AgCl$.

\paragraph*{2.2}

Quanto mais forte o ácido, maior seu $K_a$, e menor seu $pKa$. Portanto, o ácido mais forte é o $C_6H_5COOH$, com constante de ionização de $6,6 \times 10^{-5}$, e o ácido mais fraco é o $H_2S$, com constante de ionização de $1,0 \times 10^{-7}$.

\paragraph*{2.9}

A massa molar do $OH^-$ é:
\begin{gather}
    m = m_H + m_O = 1g/mol + 16g/mol = 17g/mol \\
\end{gather}

A concentração será:
\begin{gather}
    [OH^-] = \frac{m}{V \times mM} = \frac{8,5 \times 10^{-3} g}{1L \times 17g/mol} = 5 \times 10^{-4} mol/L
\end{gather}

Logo, o pOH e pH serão:
\begin{gather}
    \begin{align}
        & pOH = -\log [OH^-] = -\log(5 \times 10^{-4}) = - (\log 5 - 4 \log 10) = - (0,7 - 4) = 3,3 \\
        & pH = 14 - pOH \\
        & pH = 14 - 3,3 = 10,7
    \end{align}
\end{gather}

\paragraph*{2.10}

\section*{Cinética química}

\paragraph*{3.1}

\begin{gather}
    \begin{align*}
        & A \rightarrow \text{produto} \\
        & - \frac{d[A]}{dt} = K[A]^1 = \frac{d[\text{produto}]}{dt} \\
        & - \frac{d[A]}{[A]} = Kdt \\
        & \int_{A}^{A_0} \frac{d[A]}{A} = - \int_{0}^{t} K dt \\
        & \ln [A] {|_{A_0}^{A}} = - Kt {|_{0}^t} \\
        & \ln[A] - \ln[A_0] = - Kt \\
        & [A] = [A_0] \exp(-Kt) \\
        & \frac{[A_0]}{2} = [A_0] \exp(-K \tau_{\frac{1}{2}}) \\
        & \frac{1}{2} = \exp(-K \tau_{\frac{1}{2}}) \\
        & -\ln 2 = -K \tau_{\frac{1}{2}} \\
        & \tau_{\frac{1}{2}} = \frac{\ln 2}{K}
    \end{align*}
\end{gather}

\paragraph*{3.2}

\begin{gather}
    \begin{align}
        & 2A(g) + 2B(g) \rightarrow C(g) \\
        & \frac{-d[A]}{dt} = K_1[A]^2
    \end{align}
\end{gather}

\paragraph*{32.3}

\begin{gather}
    \begin{align}
        & A \rightarrow \text{produto} \\
        & -\frac{d[A]}{dt} = K_1 [A]^{\frac{1}{2}}
    \end{align}
\end{gather}

\paragraph*{32.4}

\begin{gather}
    \begin{align}
        & [A] = [A_0] \exp(-Kt) \\
        & \frac{[A_0]}{100} \times 32,5 = [A_0] \exp(-Kt) \\
        & \frac{32,5}{100} = \exp(-Kt) \\
        & \ln \frac{32,5}{100} = -K \times 540s \\
        & \frac{\ln 32,5 - \ln 100}{540s} = -K \\
        & \frac{3,48 - 4,61}{540} = -K \\
        & K = 2,1 \times 10^{-3} s^{-1}
    \end{align}
\end{gather}

\begin{gather}
    \begin{align}
        & [A] = [A_0] \exp(-Kt) \\
        & \frac{3}{4} [A_0] = [A_0] \exp(-Kt) \\
        & \frac{3}{4} = \exp(-Kt) \\
        & \ln 3 - \ln 4 = -Kt \\
        & -0,29 = -Kt \\
        & t = \frac{2,9 \times 10^{-1}}{2,1 \times 10^{-3}} \\
        & t = 138s
    \end{align}
\end{gather}

\paragraph*{32.5}

\begin{gather}
    \begin{align}
        & t = \frac{\ln 2}{K} \\
        & 30 \times 60 = \frac{\ln 2}{K} \\
        & K = \frac{\ln 2}{1800} = 3,85 \times 10^{-4}
    \end{align}
\end{gather}

\begin{gather}
    \begin{align}
        & [A] = [A_0] \exp(-Kt) \\
        & \frac{[A]}{[A_0]} = \exp(-Kt) \\
        & \frac{[A]}{[A_0]} = e^{1,62} \\
        & \frac{[A]}{[A_0]} = 0,198
    \end{align}
\end{gather}

\paragraph*{Reação de 2ª ordem}

\begin{gather}
    \begin{align}
        & 2[A] \rightarrow \text{produto} \\
        & -\frac{d[A]}{dt} = K_1[A]^2 \\
        & -\frac{d[A]}{[A]} = Kdt \\
        & - \int_{A_0}^{A} \frac{d[A]}{[A]^2} = K \int_{0}^{t} dt \\
        & \frac{[A]^{-1}}{-1} \bigg\rvert_{A_0}^{[A]} = -Kt \bigg\rvert_0^t \\
        & \frac{1}{[A]} \bigg\rvert_{A_0}^{[A]} = Kt \bigg\rvert_0^t \\
        & \frac{1}{[A]} - \frac{1}{[A_0]} = Kt
    \end{align}
\end{gather}

Meia vida:

\begin{gather}
    \begin{align}
        & \frac{1}{\frac{A_0}{2}} - \frac{1}{[A_0]} = K \tau \\
        & \frac{2}{[A_0]} - \frac{1}{[A_0]} = K \tau \\
        & \frac{1}{[A_0]} = K \tau \\
        & \tau = \frac{1}{K[A_0]}
    \end{align}
\end{gather}

\paragraph*{32.16}

\begin{gather}
    \begin{align}
        & \tau = \frac{1}{K[A_0]} \\
        & [A_0] = 0,05 mol/L \rightarrow \tau = \frac{1}{6,8 \times 10^{-4} \times 0,05} \\
        & \tau = 2,9 \times 10^4 s \\
        & [A_0] = 0,01 mol/L \rightarrow \tau = \frac{1}{6,8 \times 10^{-4} \times 0,01} \\
        & \tau = 1,47 \times 10^5 s
    \end{align}
\end{gather}

\paragraph*{32.17}

\begin{gather}
    \begin{align}
        & [A] + [B] \rightarrow \text{produto} \\
        & \frac{dP}{dt} = K[A][B] \\
        & \frac{dx}{dt} = K(a - x)(b - x) \\
        & \frac{dx}{(a-x)(b-x)} = Kdt \\
        & \int_{0}^{x} \frac{dx}{(a-x)(b-x)} = \int_{0}^{t} Kdt \\
        & \frac{1}{b-a} \left( \ln \left( \frac{a}{a-x} \right) - \ln \left( \frac{b}{b-x} \right)   \right) = Kt
    \end{align}
\end{gather}

\paragraph*{32.36}

Descobrir a expressão da taxa de decomposição do ozônio:

Para a decomposição do ozônio, temos que:
\begin{gather}
    \begin{align}
        & \frac{d[O_3]}{dt} = -K_1 [O_3] + K_{-1} [O] [O_2] - K_2 [O_3][O] \\
        & \frac{d[O]}{dt} = K_1 [O_3] - K_{-1} [O] [O_2] - K_2 [O_3] [O] \\
        & 0 =  K_1 [O_3] - K_{-1} [O] [O_2] - K_2 [O_3] [O] \\
        & K_1 [O_3] = K_{-1} [O] [O_2] + K_2 [O_3] [O] \\
        & K_1 [O_3] = [O] \ (K_{-1} [O_2] + K_2 [O_3]) \\
        & [O] = \frac{K_1 [O_3]}{K_{-1} [O_2] + K_2 [O_3]} \\
        & \frac{d[O_3]}{dt} = -K_1 [O_3] + (K_{-1} [O_2] - K_2 [O_3]) \frac{K_1 [O_3]}{K_{-1} [O_2] + K_2 [O_3]}
    \end{align}
\end{gather}

Quando $K_{-1} [O_2] << K_2 [O_3]$, temos:
\begin{gather}
    \begin{align}
        & \frac{d[O_3]}{dt} = -K_1 [O_3] + (K_{-1} [O_2] - K_2 [O_3]) \frac{K_1 [O_3]}{K_{-1} [O_2] + K_2 [O_3]} \\
        & \frac{d[O_3]}{dt} = -K_1[O_3] - \frac{K_2 [O_3] K_1[O_3]}{K_2[O_3]} \\
        & \frac{d[O_3]}{dt} = -2K_1 [O_3]
    \end{align}
\end{gather}