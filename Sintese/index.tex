\documentclass[a4paper]{article}

\usepackage[utf8]{inputenc}
\usepackage[T1]{fontenc}
\usepackage{textcomp}
\usepackage[dutch]{babel}
\usepackage{amsmath, amssymb}


% figure support
\usepackage{import}
\usepackage{xifthen}
\usepackage{chemfig}
\pdfminorversion=7
\usepackage{pdfpages}
\usepackage{transparent}
\newcommand{\incfig}[1]{%
  \def\svgwidth{\columnwidth}
  \import{./figures/}{#1.pdf_tex}
}

\pdfsuppresswarningpagegroup=1

\begin{document}

\setchemfig{atom style={scale=0.75}} % Ajusta o tamanho dos átomos

\section*{Reação de polimerização do metacrilato de metila}

\begin{center}
  \schemestart
      \chemfig{C(-[:90]H)(-[:-90]H)=[@{init}]O}
          \+
              \chemfig{C(-[:90]H)(-[:-90]H)(-[:0]OOC[:0]C_6H_5COO)}
                  \arrow{->[$\Delta$]}
                      \chemfig{C(-[:90]H)(-[:-90]C_4H_9)(-[:0]OOC[:0]C_6H_5COO)}
                          \+
                              \chemfig{C(-[:90]H)(-[:-90]C_6H_5)(-[:0]OOC[:0]C_4H_9)}
                              \schemestop
                            \end{center}

                            \section*{Reação com agente de transferência de cadeia (n-dodecil mercaptano)}

                            \begin{center}
                              \schemestart
                                  \chemfig{C(-[:90]H)(-[:-90]H)=[@{init}]O}
                                      \+
                                          \chemfig{C(-[:90]H)(-[:-90]H)(-[:0]OOC[:0]C_6H_5COO)}
                                              \arrow{->[$\Delta$]}
                                                  \chemfig{C(-[:90]H)(-[:-90]C_4H_9)(-[:0]OOC[:0]C_6H_5COO)}
                                                      \+
                                                          \chemfig{C(-[:90]H)(-[:-90]C_{12}H_{25})(-[:0]OOC[:0]C_4H_9)}
                                                          \schemestop
                                                        \end{center}

\end{document}
