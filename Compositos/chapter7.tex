\lecture{7}{}{Macromecânica dos materiais compósitos}

\begin{multicols*}{2}

  \begin{itemize}
    \item Lei de Hooke generalizada
    \item Tensoes normais com deformações normais no mesmo eixo - primeira metade da diagonal principal da matriz - vermelho
    \item "           " em direções perpendiculares (coeficiente de Poisson) - roxo 
    \item Tensões cisalhantes que geram deformações normais - amarelo
    \item Tensões normais que geram deformações cisalhantes - amarelo (diferente quadrante) 
    \item Módulo de cisalhamento - deformações angulare com tensões cisalhantes no mesmo plano - verde
    \item Tensões e deformações cisalhantes em planos diferentes - azul
    \item A quantidade de constantes elásticas independentes na Lei de Hooke generalizada dependerá do tipo de material (anisotrópico, isotrópico, transversalmente isotrópico, etc.)
    \item Em laminados, pode-se aproximar as tensões num eixo a zero.
  \end{itemize}

\end{multicols*}
